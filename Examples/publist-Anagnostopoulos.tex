\documentclass[a4paper,10pt]{article} 
\usepackage{amssymb} \usepackage{amsfonts}\usepackage{amsmath}
\usepackage{fontspec}\usepackage{xunicode}\usepackage{xltxtra}
\usepackage{hyperref}
\usepackage [a4paper, total={6.5in, 10in}]{geometry}  % Increase width of text https://www.sharelatex.com/learn/Page_size_and_margins
\setmainfont[Mapping=tex-text]{GFS Didot}
\hypersetup{pdftitle={Publications of Konstantinos Anagnostopoulos}, colorlinks=true,linkcolor=blue,filecolor=magenta,citecolor = blue,anchorcolor = blue, urlcolor=blue, bookmarks=true}
\begin{document}
\begin{center}
{\Large\bf Konstantinos Anagnostopoulos}\\
{\large\bf List of Publications}
\end  {center}
\begin{enumerate}
%-------------------------------------------------------------------------
% NewRecord: 1. 2307.01681 Hirasawa:2023lpb SlacID: 2674324 DOI: NoDOI Citations:  0 ( 0 ) Pages: 13
\item Mitsuaki Hirasawa, Konstantinos N. Anagnostopoulos, Takehiro Azuma, Kohta Hatakeyama, Jun Nishimura, Stratos Papadoudis, Asato Tsuchiya, {\it ``The emergence of expanding space-time in the Lorentzian type IIB matrix model with a novel regularization''}, \href{https://arxiv.org/abs/2307.01681}{arXiv:2307.01681}.

The Lorentzian type IIB matrix model is a promising candidate for a non-perturbative formulation of superstring theory. However, it was recently found that a Euclidean space-time appears in the conventional large-N limit. In this work, we study the model with a Lorentz invariant mass term which can be considered as an IR regulator. By performing complex Langevin simulations to overcome the sign problem, we observe the emergence of expanding space-time with Lorentzian signature.
%-------------------------------------------------------------------------
% NewRecord: 2. 2212.10127 Hirasawa:2022qzg SlacID: 2616279 DOI: 10.22323/1.430.0371 Citations:  2 ( 2 ) Pages: 10
\item Mitsuaki Hirasawa, Konstantinos N. Anagnostopoulos, Takehiro Azuma, Kohta Hatakeyama, Jun Nishimura, Stratos Kovalkov Papadoudis, Asato Tsuchiya, {\it ``The emergence of expanding space-time in a novel large-$N$ limit of the Lorentzian type IIB matrix model''}, \href{https://www.doi.org/10.22323/1.430.0371}{PoS {\bf LATTICE2022} (2023) 371} \href{https://arxiv.org/abs/2212.10127}{[arXiv:2212.10127]}, \href{https://www.doi.org/10.22323/1.430.0371}{[doi:10.22323/1.430.0371]}.
\\\href{https://inspirehep.net/literature/?q=refersto%3Arecid%3A2616279}{Cited by 2 (2)} articles

The Lorentzian type IIB matrix model is a promising candidate for a non-perturbative formulation of superstring theory. However, it was found recently that a Euclidean space-time appears in the conventional large-N limit. In this work, we add a Lorentz invariant mass term to the original model and consider a limit, in which the coefficient of the mass term vanishes at large N. By performing complex Langevin simulations to overcome the sign problem, we observe the emergence of expanding space-time with the Lorentzian signature.
%-------------------------------------------------------------------------
% NewRecord: 3. 2210.17537 Anagnostopoulos:2022dak SlacID: 2173460 DOI: 10.1140/epjs/s11734-023-00849-x Citations:  4 ( 1 ) Pages: 23
\item Konstantinos N. Anagnostopoulos, Takehiro Azuma, Kohta Hatakeyama, Mitsuaki Hirasawa, Yuta Ito, Jun Nishimura, Stratos Kovalkov Papadoudis, Asato Tsuchiya, {\it ``Progress in the numerical studies of the type IIB matrix model''}, \href{https://arxiv.org/abs/2210.17537}{arXiv:2210.17537}, \href{https://www.doi.org/10.1140/epjs/s11734-023-00849-x}{[doi:10.1140/epjs/s11734-023-00849-x]}.
\\\href{https://inspirehep.net/literature/?q=refersto%3Arecid%3A2173460}{Cited by 4 (1)} articles

The type IIB matrix model, also known as the IKKT model, has been proposed as a promising candidate for a non-perturbative formulation of superstring theory. Based on this proposal, various attempts have been made to explain how our four-dimensional space-time can emerge dynamically from superstring theory. In this article, we review the progress in numerical studies on the type IIB matrix model. We particularly focus on the most recent results for the Euclidean and Lorentzian versions, which are obtained using the complex Langevin method to overcome the sign problem. We also review the earlier results obtained using conventional Monte Carlo methods and clarify the relationship among different calculations.
%-------------------------------------------------------------------------
% NewRecord: 4. 2201.13200 Hatakeyama:2022ybs SlacID: 2022364 DOI: 10.1142/9789811261633_0002 Citations:  7 ( 4 ) Pages: 10
\item Kohta Hatakeyama, Konstantinos Anagnostopoulos, Takehiro Azuma, Mitsuaki Hirasawa, Yuta Ito, Jun Nishimura, Stratos Papadoudis, Asato Tsuchiya, {\it ``Complex Langevin studies of the emergent space-time in the type IIB matrix model''}, \href{https://arxiv.org/abs/2201.13200}{arXiv:2201.13200}, \href{https://www.doi.org/10.1142/9789811261633_0002}{[doi:10.1142/9789811261633\_0002]}.
\\\href{https://inspirehep.net/literature/?q=refersto%3Arecid%3A2022364}{Cited by 7 (4)} articles

The type IIB matrix model has been proposed as a non-perturbative definition of superstring theory since 1996. We study a simplified model that describes the late time behavior of the type IIB matrix model non-perturbatively using Monte Carlo methods, and we use the complex Langevin method to overcome the sign problem. We investigate a scenario where the space-time signature changes dynamically from Euclidean at early times to Lorentzian at late times. We discuss the possibility of the emergence of the (3+1)D expanding universe.
%-------------------------------------------------------------------------
% NewRecord: 5. 2112.15368 Hatakeyama:2021ake SlacID: 1998854 DOI: 10.22323/1.396.0341 Citations:  11 ( 6 ) Pages: 12
\item Kohta Hatakeyama, Konstantinos Anagnostopoulos, Takehiro Azuma, Mitsuaki Hirasawa, Yuta Ito, Jun Nishimura, Stratos Papadoudis, Asato Tsuchiya, {\it ``Relationship between the Euclidean and Lorentzian versions of the type IIB matrix model''}, \href{https://www.doi.org/10.22323/1.396.0341}{PoS {\bf LATTICE2021} (2022) 341} \href{https://arxiv.org/abs/2112.15368}{[arXiv:2112.15368]}, \href{https://www.doi.org/10.22323/1.396.0341}{[doi:10.22323/1.396.0341]}.
\\\href{https://inspirehep.net/literature/?q=refersto%3Arecid%3A1998854}{Cited by 11 (6)} articles

The type IIB matrix model was proposed as a non-perturbative formulation of superstring theory in 1996. We simulate a model that describes the late time behavior of the IIB matrix model by applying the complex Langevin method to overcome the sign problem. We clarify the relationship between the Euclidean and the Lorentzian versions of the type IIB matrix model in a recently discovered phase. By introducing a constraint, we obtain a model where the spacetime metric is Euclidean at early times, whereas it dynamically becomes Lorentzian at late times.
%-------------------------------------------------------------------------
% NewRecord: 6. 2112.15390 Hirasawa:2021xeh SlacID: 1998865 DOI: 10.22323/1.396.0428 Citations:  9 ( 5 ) Pages: 10
\item Mitsuaki Hirasawa, Konstantinos Anagnostopoulos, Takehiro Azuma, Kohta Hatakeyama, Yuta Ito, Jun Nishimura, Stratos Papadoudis, Asato Tsuchiya, {\it ``A new phase in the Lorentzian type IIB matrix model and the emergence of continuous space-time''}, \href{https://www.doi.org/10.22323/1.396.0428}{PoS {\bf LATTICE2021} (2022) 428} \href{https://arxiv.org/abs/2112.15390}{[arXiv:2112.15390]}, \href{https://www.doi.org/10.22323/1.396.0428}{[doi:10.22323/1.396.0428]}.
\\\href{https://inspirehep.net/literature/?q=refersto%3Arecid%3A1998865}{Cited by 9 (5)} articles

The Lorentzian type IIB matrix model is a promising candidate for a non-perturbative formulation of superstring theory. In previous studies, Monte Carlo calculations provided interesting results indicating the spontaneous breaking of SO(9) to SO(3) and the emergence of (3+1)-dimensional space-time. However, an approximation was used to avoid the sign problem, which seemed to make the space-time structure singular. In this talk, we report our results obtained by using the complex Langevin method to overcome the sign problem instead of using this approximation. In particular, we discuss the emergence of continuous space-time in a new phase, which we discovered recently.
%-------------------------------------------------------------------------
% NewRecord: 7. 2009.08682 Anagnostopoulos:2020cwo SlacID: 1818095 DOI: NoDOI Citations:  0 ( 0 ) Pages: 5
\item Konstantinos N. Anagnostopoulos, Takehiro Azuma, Yuta Ito, Jun Nishimura, Toshiyuki Okubo, Stratos Kovalkov Papadoudis, {\it ``Complex Langevin studies of the dynamical compactification of extra dimensions in the Euclidean IKKT matrix model''}, \href{https://arxiv.org/abs/2009.08682}{arXiv:2009.08682}.

The type IIB matrix model, also known as the IKKT matrix model, is a promising candidate for a nonperturbative formulation of superstring theory. In this talk we study the Euclidean version of the IKKT matrix model, which has a "sign problem" due to the Pfaffian coming from integrating out the fermionic degrees of freedom. To study the spontaneous breaking of the SO(10) rotational symmetry, we apply the Complex Langevin Method (CLM) to the Euclidean IKKT matrix model. We conclude that the SO(10) symmetry is broken to SO(3), in agreement with the previous studies by the Gaussian Expansion Method (GEM). We also apply the GEM to the deformed model and find consistency with the CLM result. These are proceedings of Takehiro Azuma's talk at Asia-Pacific Symposium for Lattice Field Theory (APLAT 2020) on August 4-7, 2020, based on the paper arXiv:2002.07410.
%-------------------------------------------------------------------------
% NewRecord: 8. 2005.12567 Anagnostopoulos:2020ebo SlacID: 1797951 DOI: 10.22323/1.376.0183 Citations:  1 ( 1 ) Pages: 15
\item Konstantinos N. Anagnostopoulos, Takehiro Azuma, Yuta Ito, Jun Nishimura, Toshiyuki Okubo, Stratos Kovalkov Papadoudis, {\it ``Dynamical Compactification of Extra Dimensions in the Euclidean IKKT Matrix Model via Spontaneous Symmetry Breaking''}, \href{https://www.doi.org/10.22323/1.376.0183}{PoS {\bf CORFU2019} (2020) 183} \href{https://arxiv.org/abs/2005.12567}{[arXiv:2005.12567]}, \href{https://www.doi.org/10.22323/1.376.0183}{[doi:10.22323/1.376.0183]}.
\\\href{https://inspirehep.net/literature/?q=refersto%3Arecid%3A1797951}{Cited by 1 (1)} articles

The IKKT matrix model has been conjectured to provide a promising nonperturbative formulation of superstring theory. In this model, spacetime emerges dynamically from the microscopic matrix degrees of freedom in the large-N limit, and Monte Carlo simulations of the Lorentzian version provide evidence of an emergent (3+1)-dimensional expanding space-time. In this talk, we discuss the Euclidean version of the IKKT matrix model and provide evidence of dynamical compactification of the extra dimensions via the spontaneous symmetry breaking (SSB) of the 10D rotational symmetry. We perform numerical simulations of a system with a severe complex action problem by using the complex Langevin method (CLM). The CLM suffers from the singular-drift problem and we deform the model in order to avoid it. We study the SSB pattern as we vary the deformation parameter and we conclude that the original model has an SO(3) symmetric vacuum, in agreement with previous calculations using the Gaussian expansion method (GEM). We employ the GEM to the deformed model and we obtain results consistent with the ones obtained by CLM.
%-------------------------------------------------------------------------
% NewRecord: 9. 2002.07410 Anagnostopoulos:2020xai SlacID: 1781312 DOI: 10.1007/JHEP06(2020)069 Citations:  25 ( 12 ) Pages: 36
\item Konstantinos N. Anagnostopoulos, Takehiro Azuma, Yuta Ito, Jun Nishimura, Toshiyuki Okubo, Stratos Kovalkov Papadoudis, {\it ``Complex Langevin analysis of the spontaneous breaking of 10D rotational symmetry in the Euclidean IKKT matrix model''}, \href{https://www.doi.org/10.1007/JHEP06(2020)069}{JHEP {\bf 06} (2020) 069} \href{https://arxiv.org/abs/2002.07410}{[arXiv:2002.07410]}, \href{https://www.doi.org/10.1007/JHEP06(2020)069}{[doi:10.1007/JHEP06(2020)069]}.
\\\href{https://inspirehep.net/literature/?q=refersto%3Arecid%3A1781312}{Cited by 25 (12)} articles

The IKKT matrix model is a promising candidate for a nonperturbative formulation of superstring theory. In this model, spacetime is conjectured to emerge dynamically from the microscopic matrix degrees of freedom in the large-N limit. Indeed in the Lorentzian version, Monte Carlo studies suggested the emergence of (3+1)-dimensional expanding spacetime. Here we study the Euclidean version instead, and investigate an alternative scenario for dynamical compactification of extra dimensions via the spontaneous symmetry breaking (SSB) of 10D rotational symmetry. We perform numerical simulations based on the complex Langevin method (CLM) in order to avoid a severe sign problem. Furthermore, in order to avoid the singular-drift problem in the CLM, we deform the model and determine the SSB pattern as we vary the deformation parameter. From these results, we conclude that the original model has an SO(3) symmetric vacuum, which is consistent with previous results obtained by the Gaussian expansion method (GEM). We also apply the GEM to the deformed matrix model and find consistency with the results obtained by the CLM.
%-------------------------------------------------------------------------
% NewRecord: 10. 1906.01841 Anagnostopoulos:2019ptt SlacID: 1738534 DOI: 10.22323/1.347.0065 Citations:  4 ( 3 ) Pages: 13
\item Konstantinos N. Anagnostopoulos, Takehiro Azuma, Yuta Ito, Jun Nishimura, Stratos Kovalkov Papadoudis, {\it ``Dynamical compactification of extra dimensions in the Euclidean type IIB matrix model: A numerical study using the complex Langevin method''}, \href{https://www.doi.org/10.22323/1.347.0065}{PoS {\bf CORFU2018} (2019) 065} \href{https://arxiv.org/abs/1906.01841}{[arXiv:1906.01841]}, \href{https://www.doi.org/10.22323/1.347.0065}{[doi:10.22323/1.347.0065]}.
\\\href{https://inspirehep.net/literature/?q=refersto%3Arecid%3A1738534}{Cited by 4 (3)} articles

The type IIB matrix model is conjectured to be a nonperturbative definition of type IIB superstring theory. In this model, spacetime is a dynamical quantity and compactification of extra dimensions can be realized via spontaneous symmetry breaking (SSB). In this work, we consider a simpler, related, six dimensional model in its Euclidean version and study it numerically. Our calculations provide evidence that the SO(6) rotational symmetry of the model breaks down to SO(3), which means that the theory lives in a vacuum where 3 out of the 6 dimensions are large compared to the other 3. Our results show the same SSB pattern predicted by the Gaussian expansion method and that they are in quantitative agreement. The Monte Carlo simulations are hindered by a severe complex action problem which is addressed by applying the complex Langevin method.
%-------------------------------------------------------------------------
% NewRecord: 11. NoArXiv Proceedings:2019szv SlacID: 1756030 DOI: NoDOI Citations:  0 ( 0 ) Pages: 
\item Konstantinos Anagnostopoulos, David Berman, Athanasios Chatzistavrakidis, Dumitru Ghilencea, Fotios Diakonos, Jan Kalinowski, Athanasios Kapoyannis, Margarida Nesbitt, Antonios Tsapalis, Dimitris Varouchas, George Zoupanos, {\it ``Proceedings, 18th Hellenic School and Workshops on Elementary Particle Physics and Gravity (CORFU2018)''}, PoS {\bf CORFU2018} (2019) .


%-------------------------------------------------------------------------
% NewRecord: 12. 1712.07562 Anagnostopoulos:2017gos SlacID: 1644379 DOI: 10.1007/JHEP02(2018)151 Citations:  29 ( 12 ) Pages: 19
\item Konstantinos N. Anagnostopoulos, Takehiro Azuma, Yuta Ito, Jun Nishimura, Stratos Kovalkov Papadoudis, {\it ``Complex Langevin analysis of the spontaneous symmetry breaking in dimensionally reduced super Yang-Mills models''}, \href{https://www.doi.org/10.1007/JHEP02(2018)151}{JHEP {\bf 02} (2018) 151} \href{https://arxiv.org/abs/1712.07562}{[arXiv:1712.07562]}, \href{https://www.doi.org/10.1007/JHEP02(2018)151}{[doi:10.1007/JHEP02(2018)151]}.
\\\href{https://inspirehep.net/literature/?q=refersto%3Arecid%3A1644379}{Cited by 29 (12)} articles

In recent years the complex Langevin method (CLM) has proven a powerful method in studying statistical systems which suffer from the sign problem. Here we show that it can also be applied to an important problem concerning why we live in four-dimensional spacetime. Our target system is the type IIB matrix model, which is conjectured to be a nonperturbative definition of type IIB superstring theory in ten dimensions. The fermion determinant of the model becomes complex upon Euclideanization, which causes a severe sign problem in its Monte Carlo studies. It is speculated that the phase of the fermion determinant actually induces the spontaneous breaking of the SO(10) rotational symmetry, which has direct consequences on the aforementioned question. In this paper, we apply the CLM to the 6D version of the type IIB matrix model and show clear evidence that the SO(6) symmetry is broken down to SO(3). Our results are consistent with those obtained previously by the Gaussian expansion method.
%-------------------------------------------------------------------------
% NewRecord: 13. 1509.05079 Anagnostopoulos:2015gua SlacID: 1393762 DOI: 10.22323/1.251.0307 Citations:  12 ( 1 ) Pages: 7
\item Konstantinos N. Anagnostopoulos, Takehiro Azuma, Jun Nishimura, {\it ``Monte Carlo studies of dynamical compactification of extra dimensions in a model of nonperturbative string theory''}, \href{https://www.doi.org/10.22323/1.251.0307}{PoS {\bf LATTICE2015} (2016) 307} \href{https://arxiv.org/abs/1509.05079}{[arXiv:1509.05079]}, \href{https://www.doi.org/10.22323/1.251.0307}{[doi:10.22323/1.251.0307]}.
\\\href{https://inspirehep.net/literature/?q=refersto%3Arecid%3A1393762}{Cited by 12 (1)} articles

The IIB matrix model has been proposed as a non-perturbative definition of superstring theory. In this work, we study the Euclidean version of this model in which extra dimensions can be dynamically compactified if a scenario of spontaneously breaking the SO(10) rotational symmetry is realized. Monte Carlo calculations of the Euclidean IIB matrix model suffer from a very strong complex action problem due to the large fluctuations of the complex phase of the Pfaffian which appears after integrating out the fermions. We employ the factorization method in order to achieve effective sampling. We report on preliminary results that can be compared with previous studies of the rotational symmetry breakdown using the Gaussian expansion method.
%-------------------------------------------------------------------------
% NewRecord: 14. NoArXiv Anagnostopoulos:2014vha SlacID: 1323135 DOI: 10.1002/prop.v62.9/10 Citations:  1 ( 1 ) Pages: 
\item Konstantinos Anagnostopoulos, Olaf Lechtenfeld, Dieter Lüst, Hikaru Kawai, Jun Nishimura, Harold Steinacker, Richard Szabo, Satoshi Watamura, George Zoupanos, {\it ``Proceedings, 13th Hellenic School and Workshops on Elementary Particle Physics and Gravity : Workshop on Noncommutative Field Theory and Gravity (CORFU2013-NC)''}, \href{https://www.doi.org/10.1002/prop.v62.9/10}{Fortsch.Phys. {\bf 62} (2014) }, \href{https://www.doi.org/10.1002/prop.v62.9/10}{[doi:10.1002/prop.v62.9/10]}.
\\\href{https://inspirehep.net/literature/?q=refersto%3Arecid%3A1323135}{Cited by 1 (1)} articles


%-------------------------------------------------------------------------
% NewRecord: 15. 1306.6135 Anagnostopoulos:2013xga SlacID: 1239963 DOI: 10.1007/JHEP11(2013)009 Citations:  30 ( 7 ) Pages: 28
\item Konstantinos N. Anagnostopoulos, Takehiro Azuma, Jun Nishimura, {\it ``Monte Carlo studies of the spontaneous rotational symmetry breaking in dimensionally reduced super Yang-Mills models''}, \href{https://www.doi.org/10.1007/JHEP11(2013)009}{JHEP {\bf 11} (2013) 009} \href{https://arxiv.org/abs/1306.6135}{[arXiv:1306.6135]}, \href{https://www.doi.org/10.1007/JHEP11(2013)009}{[doi:10.1007/JHEP11(2013)009]}.
\\\href{https://inspirehep.net/literature/?q=refersto%3Arecid%3A1239963}{Cited by 30 (7)} articles

It has long been speculated that the spontaneous symmetry breaking (SSB) of SO(D) occurs in matrix models obtained by dimensionally reducing super Yang-Mills theory in D=6,10 dimensions. In particular, the D=10 case corresponds to the IIB matrix model, which was proposed as a nonperturbative formulation of superstring theory, and the SSB may correspond to the dynamical generation of four-dimensional space-time. Recently, it has been shown by using the Gaussian expansion method that the SSB indeed occurs for D=6 and D=10, and interesting nature of the SSB common to both cases has been suggested. Here we study the same issue from first principles by a Monte Carlo method in the D=6 case. In spite of a severe complex-action problem, the factorization method enables us to obtain various quantities associated with the SSB, which turn out to be consistent with the previous results obtained by the Gaussian expansion method. This also demonstrates the usefulness of the factorization method as a general approach to systems with the complex-action problem or the sign problem.
%-------------------------------------------------------------------------
% NewRecord: 16. 1211.0950 Anagnostopoulos:2012ib SlacID: 1198171 DOI: 10.22323/1.164.0226 Citations:  3 ( 2 ) Pages: 7
\item Konstantinos N. Anagnostopoulos, Takehiro Azuma, Jun Nishimura, {\it ``Monte Carlo Simulations of a Supersymmetric Matrix Model of Dynamical Compactification in Non Perturbative String Theory''}, \href{https://www.doi.org/10.22323/1.164.0226}{PoS {\bf LATTICE2012} (2012) 226} \href{https://arxiv.org/abs/1211.0950}{[arXiv:1211.0950]}, \href{https://www.doi.org/10.22323/1.164.0226}{[doi:10.22323/1.164.0226]}.
\\\href{https://inspirehep.net/literature/?q=refersto%3Arecid%3A1198171}{Cited by 3 (2)} articles

The IKKT or IIB matrix model has been postulated to be a non perturbative definition of superstring theory. It has the attractive feature that spacetime is dynamically generated, which makes possible the scenario of dynamical compactification of extra dimensions, which in the Euclidean model manifests by spontaneously breaking the SO(10) rotational invariance (SSB). In this work we study using Monte Carlo simulations the 6 dimensional version of the Euclidean IIB matrix model. Simulations are found to be plagued by a strong complex action problem and the factorization method is used for effective sampling and computing expectation values of the extent of spacetime in various dimensions. Our results are consistent with calculations using the Gaussian Expansion method which predict SSB to SO(3) symmetric vacua, a finite universal extent of the compactified dimensions and finite spacetime volume.
%-------------------------------------------------------------------------
% NewRecord: 17. NoArXiv Anagnostopoulos:2012uga SlacID: 1249424 DOI: NoDOI Citations:  0 ( 0 ) Pages: 
\item K. Anagnostopoulos, I. Bakas, Nikos Irges, J. Kalinowski, Alex Kehagias, Roberto Pittau, M. Nesbitt Rebelo, G. Wolschin, G. Zoupanos, {\it ``Proceedings, 12th Hellenic School and Workshops on Elementary Particle Physics and Gravity (CORFU2012-ST), (CORFU2012-PU) and (CORFU2012-SM)''}, PoS {\bf CORFU2012} (2012) .


%-------------------------------------------------------------------------
% NewRecord: 18. 1110.6531 Anagnostopoulos:2011rr SlacID: 943733 DOI: 10.22323/1.139.0181 Citations:  2 ( 1 ) Pages: 7
\item Konstantinos N. Anagnostopoulos, Takehiro Azuma, Jun Nishimura, {\it ``Towards an Effective Importance Sampling in Monte Carlo Simulations of a System with a Complex Action''}, \href{https://www.doi.org/10.22323/1.139.0181}{PoS {\bf LATTICE2011} (2011) 181} \href{https://arxiv.org/abs/1110.6531}{[arXiv:1110.6531]}, \href{https://www.doi.org/10.22323/1.139.0181}{[doi:10.22323/1.139.0181]}.
\\\href{https://inspirehep.net/literature/?q=refersto%3Arecid%3A943733}{Cited by 2 (1)} articles

The sign problem is a notorious problem, which occurs in Monte Carlo simulations of a system with a partition function whose integrand is not positive. One way to simulate such a system is to use the factorization method where one enforces sampling in the part of the configuration space which gives important contribution to the partition function. This is accomplished by using constraints on some observables chosen appropriately and minimizing the free energy associated with their joint distribution functions. These observables are maximally correlated with the complex phase. Observables not in this set essentially decouple from the phase and can be calculated without the sign problem in the corresponding 'microcanonical' ensemble. These ideas are applied on a simple matrix model with very strong sign problem and the results are found to be consistent with analytic calculations using the Gaussian Expansion Method.
%-------------------------------------------------------------------------
% NewRecord: 19. 1108.1534 Anagnostopoulos:2011cn SlacID: 922419 DOI: 10.1007/JHEP10(2011)126 Citations:  21 ( 3 ) Pages: 44
\item Konstantinos N. Anagnostopoulos, Takehiro Azuma, Jun Nishimura, {\it ``A practical solution to the sign problem in a matrix model for dynamical compactification''}, \href{https://www.doi.org/10.1007/JHEP10(2011)126}{JHEP {\bf 10} (2011) 126} \href{https://arxiv.org/abs/1108.1534}{[arXiv:1108.1534]}, \href{https://www.doi.org/10.1007/JHEP10(2011)126}{[doi:10.1007/JHEP10(2011)126]}.
\\\href{https://inspirehep.net/literature/?q=refersto%3Arecid%3A922419}{Cited by 21 (3)} articles

The matrix model formulation of superstring theory offers the possibility to understand the appearance of 4d space-time from 10d as a consequence of spontaneous breaking of the SO(10) symmetry. Monte Carlo studies of this issue is technically difficult due to the so-called sign problem. We present a practical solution to this problem generalizing the factorization method proposed originally by two of the authors (K.N.A. and J.N.). Explicit Monte Carlo calculations and large-N extrapolations are performed in a simpler matrix model with similar properties, and reproduce quantitative results obtained previously by the Gaussian expansion method. Our results also confirm that the spontaneous symmetry breaking indeed occurs due to the phase of the fermion determinant, which vanishes for collapsed configurations. We clarify various generic features of this approach, which would be useful in applying it to other statistical systems with the sign problem.
%-------------------------------------------------------------------------
% NewRecord: 20. NoArXiv Anagnostopoulos:2011sgb SlacID: 1197783 DOI: NoDOI Citations:  0 ( 0 ) Pages: 
\item K. Anagnostopoulos, I. Antoniadis, D. Bahns, N. Irges, A. Kehagias, G. Lazarides, Dieter Luest, Harold Steinacker, George Zoupanos, {\it ``Proceedings, 11th Hellenic School and Workshops on Elementary Particle Physics and Gravity (CORFU2011)''}, PoS {\bf CORFU2011} (2011) .


%-------------------------------------------------------------------------
% NewRecord: 21. NoArXiv Anagnostopoulos:2011du SlacID: 1197801 DOI: NoDOI Citations:  0 ( 0 ) Pages: 170
\item Konstantinos Anagnostopoulos, Nikos Irges, George Zoupanos, {\it ``Elementary particle physics and gravity. Proceedings, 10th Hellenic Schools and Workshops, Corfu 2010, Corfu, August 29-September 12, 2010''}, Fortsch.Phys. {\bf 59} (2011) .


%-------------------------------------------------------------------------
% NewRecord: 22. 1009.4504 Anagnostopoulos:2010ux SlacID: 870877 DOI: 10.1103/PhysRevD.83.054504 Citations:  26 ( 9 ) Pages: 4
\item Konstantinos N. Anagnostopoulos, Takehiro Azuma, Jun Nishimura, {\it ``A General approach to the sign problem: The Factorization method with multiple observables''}, \href{https://www.doi.org/10.1103/PhysRevD.83.054504}{Phys.Rev.D {\bf 83} (2011) 054504} \href{https://arxiv.org/abs/1009.4504}{[arXiv:1009.4504]}, \href{https://www.doi.org/10.1103/PhysRevD.83.054504}{[doi:10.1103/PhysRevD.83.054504]}.
\\\href{https://inspirehep.net/literature/?q=refersto%3Arecid%3A870877}{Cited by 26 (9)} articles

The sign problem is a notorious problem, which occurs in Monte Carlo simulations of a system with the partition function whose integrand is not real positive. The basic idea of the factorization method applied on such a system is to control some observables in order to determine and sample efficiently the region of configuration space which gives important contribution to the partition function. We argue that it is crucial to choose appropriately the set of the observables to be controlled in order for the method to work successfully in a general system. This is demonstrated by an explicit example, in which it turns out to be necessary to control more than one observables. Extrapolation to large system size is possible due to the nice scaling properties of the factorized functions, and known results obtained by an analytic method are shown to be consistently reproduced.
%-------------------------------------------------------------------------
% NewRecord: 23. 1010.0957 Anagnostopoulos:2010yu SlacID: 871968 DOI: 10.22323/1.105.0167 Citations:  3 ( 0 ) Pages: 7
\item Konstantinos N. Anagnostopoulos, Takehiro Azuma, Jun Nishimura, {\it ``A Study of the Complex Action Problem in a Simple Model for Dynamical Compactification in Superstring Theory Using the Factorization Method''}, \href{https://www.doi.org/10.22323/1.105.0167}{PoS {\bf LATTICE2010} (2010) 167} \href{https://arxiv.org/abs/1010.0957}{[arXiv:1010.0957]}, \href{https://www.doi.org/10.22323/1.105.0167}{[doi:10.22323/1.105.0167]}.
\\\href{https://inspirehep.net/literature/?q=refersto%3Arecid%3A871968}{Cited by 3 (0)} articles

The IIB matrix model proposes a mechanism for dynamically generating four dimensional space--time in string theory by spontaneous breaking of the ten dimensional rotational symmetry textrm{SO}(10). Calculations using the Gaussian expansion method (GEM) lend support to this conjecture. We study a simple textrm{SO}(4) invariant matrix model using Monte Carlo simulations and we confirm that its rotational symmetry breaks down, showing that lower dimensional configurations dominate the path integral. The model has a strong complex action problem and the calculations were made possible by the use of the factorization method on the density of states rhon(x) of properly normalized eigenvalues tildelambdan of the space--time moment of inertia tensor. We study scaling properties of the factorized terms of rhon(x) and we find them in agreement with simple scaling arguments. These can be used in the finite size scaling extrapolation and in the study of the region of configuration space obscured by the large fluctuations of the phase. The computed values of tildelambdan are in reasonable agreement with GEM calculations and a numerical method for comparing the free energy of the corresponding ansatze is proposed and tested.
%-------------------------------------------------------------------------
% NewRecord: 24. 1007.0355 Anagnostopoulos:2010gw SlacID: 860357 DOI: NoDOI Citations:  20 ( 16 ) Pages: 23
\item K. Anagnostopoulos, K. Farakos, P. Pasipoularides, A. Tsapalis, {\it ``Non-Linear Sigma Model and asymptotic freedom at the Lifshitz point''}, \href{https://arxiv.org/abs/1007.0355}{arXiv:1007.0355}.
\\\href{https://inspirehep.net/literature/?q=refersto%3Arecid%3A860357}{Cited by 20 (16)} articles

We construct the general O(N)-symmetric non-linear sigma model in 2+1 spacetime dimensions at the Lifshitz point with dynamical critical exponent z=2. For a particular choice of the free parameters, the model is asymptotically free with the beta function coinciding to the one for the conventional sigma model in 1+1 dimensions. In this case, the model admits also a simple description in terms of adjoint currents.
%-------------------------------------------------------------------------
% NewRecord: 25. NoArXiv Anagnostopoulos:2010zza SlacID: 878011 DOI: NoDOI Citations:  0 ( 0 ) Pages: 310
\item Konstantinos Anagnostopoulos, George Zoupanos, {\it ``Elementary particle physics and gravity. Proceedings, Corfu Summer Institute, School and Workshops on 'Standard model and beyond and standard cosmology' and on 'Cosmology and strings: Theory - cosmology - phenomenology', CORFU 2009, Corfu, Greece, August 30-September 13, 2009''}.


%-------------------------------------------------------------------------
% NewRecord: 26. NoArXiv Anagnostopoulos:2010zz SlacID: 913956 DOI: NoDOI Citations:  0 ( 0 ) Pages: 
\item Konstantinos N. Anagnostopoulos, Dorothea Bahns, Harald Grosse, Nikos Irges, George Zoupanos, {\it ``Proceedings, Satellite Workshop on Non Commutative Field Theory and Gravity : 10th Hellenic School and Workshops on Elementary Particle Physics and Gravity (CORFU2010-NC)''}, PoS {\bf CNCFG2010} (2010) .

It is generally expected that due to the interplay between gravity and quantum mechanics, the classical picture of smooth spacetime manifolds should be replaced by some kind of quantum geometry at very short scales. Much research has been devoted to this question in the past from various points of view. In particular, the study of quantum field theory on non commutative spacetimes has been developed as an effective approach to study physical models on quantum spaces. The workshop focused mainly on various aspects of field theories defined on non commutative spaces and on the relation among quantum , non commutative geometries and gravity in different approaches.
%-------------------------------------------------------------------------
% NewRecord: 27. 0806.3506 Ambjorn:2008jg SlacID: 788863 DOI: 10.1016/j.nuclphysb.2008.08.030 Citations:  32 ( 18 ) Pages: 19
\item J. Ambjorn, K.N. Anagnostopoulos, R. Loll, I. Pushkina, {\it ``Shaken, but not stirred: Potts model coupled to quantum gravity''}, \href{https://www.doi.org/10.1016/j.nuclphysb.2008.08.030}{Nucl.Phys.B {\bf 807} (2009) } \href{https://arxiv.org/abs/0806.3506}{[arXiv:0806.3506]}, \href{https://www.doi.org/10.1016/j.nuclphysb.2008.08.030}{[doi:10.1016/j.nuclphysb.2008.08.030]}.
\\\href{https://inspirehep.net/literature/?q=refersto%3Arecid%3A788863}{Cited by 32 (18)} articles

We investigate the critical behaviour of both matter and geometry of the three-state Potts model coupled to two-dimensional Lorentzian quantum gravity in the framework of causal dynamical triangulations. Contrary to what general arguments on the effects of disorder suggest, we find strong numerical evidence that the critical exponents of the matter are not changed under the influence of quantum fluctuations in the geometry, compared to their values on fixed, regular lattices. This lends further support to previous findings that quantum gravity models based on causal dynamical triangulations are in many ways better behaved than their Euclidean counterparts.
%-------------------------------------------------------------------------
% NewRecord: 28. 0707.4454 Anagnostopoulos:2007fw SlacID: 756942 DOI: 10.1103/PhysRevLett.100.021601 Citations:  215 ( 151 ) Pages: 4
\item Konstantinos N. Anagnostopoulos, Masanori Hanada, Jun Nishimura, Shingo Takeuchi, {\it ``Monte Carlo studies of supersymmetric matrix quantum mechanics with sixteen supercharges at finite temperature''}, \href{https://www.doi.org/10.1103/PhysRevLett.100.021601}{Phys.Rev.Lett. {\bf 100} (2008) 021601} \href{https://arxiv.org/abs/0707.4454}{[arXiv:0707.4454]}, \href{https://www.doi.org/10.1103/PhysRevLett.100.021601}{[doi:10.1103/PhysRevLett.100.021601]}.
\\\href{https://inspirehep.net/literature/?q=refersto%3Arecid%3A756942}{Cited by 215 (151)} articles

We present the first Monte Carlo results for supersymmetric matrix quantum mechanics with sixteen supercharges at finite temperature. The recently proposed non-lattice simulation enables us to include the effects of fermionic matrices in a transparent and reliable manner. The internal energy nicely interpolates the weak coupling behavior obtained by the high temperature expansion, and the strong coupling behavior predicted from the dual black hole geometry. The Polyakov line takes large values even at low temperature suggesting the absence of a phase transition in sharp contrast to the bosonic case. These results provide highly non-trivial evidences for the gauge/gravity duality.
%-------------------------------------------------------------------------
% NewRecord: 29. 0801.4205 Nishimura:2007clg SlacID: 778236 DOI: 10.22323/1.042.0059 Citations:  9 ( 6 ) Pages: 7
\item Jun Nishimura, Konstantinos N. Anagnostopoulos, Masanori Hanada, Shingo Takeuchi, {\it ``Putting M theory on a computer''}, \href{https://www.doi.org/10.22323/1.042.0059}{PoS {\bf LATTICE2007} (2007) 059} \href{https://arxiv.org/abs/0801.4205}{[arXiv:0801.4205]}, \href{https://www.doi.org/10.22323/1.042.0059}{[doi:10.22323/1.042.0059]}.
\\\href{https://inspirehep.net/literature/?q=refersto%3Arecid%3A778236}{Cited by 9 (6)} articles

We propose a non-lattice simulation for studying supersymmetric matrix quantum mechanics in a non-perturbative manner. In particular, our method enables us to put M theory on a computer based on its matrix formulation proposed by Banks, Fischler, Shenker and Susskind. Here we present Monte Carlo results of the same matrix model but in a different parameter region, which corresponds to the 't Hooft large-N limit at finite temperature. In the strong coupling limit the model has a dual description in terms of the N D0-brane solution in 10d type IIA supergravity. Our results provide highly nontrivial evidences for the conjectured duality. In particular, the energy (and hence the entropy) of the non-extremal black hole has been reproduced by solving directly the strongly coupled dynamics of the D0-brane effective theory.
%-------------------------------------------------------------------------
% NewRecord: 30. NoArXiv Anagnostopoulos:2006yg SlacID: 736740 DOI: 10.1088/1742-6596/53/1/E01 Citations:  0 ( 0 ) Pages: 899
\item K. Anagnostopoulos, I. Antoniadis, G. Fanourakis, A. Kehagias, A. Savoy-Navarro, J. Wess, G. Zoupanos, {\it ``Elementary particle physics. Proceedings, Corfu Summer Institute, CORFU2005, Corfu, Greece, September 4-26, 2005''}, \href{https://www.doi.org/10.1088/1742-6596/53/1/E01}{[doi:10.1088/1742-6596/53/1/E01]}.

These are the Proceedings of the Corfu Summer Institute on Elementary Particle Physics (CORFU2005) (http://corfu2005.physics.uoi.gr), which took place in Corfu, Greece from 4–26 September 2005. The Corfu Summer Institute has a very long, interesting and successful history, some elements of which can be found inhttp://www.corfu-summer-institute.gr.In short, the Corfu Meeting started as a Summer School on Elementary Particle Physics (EPP) mostly for Greek graduate students in 1982 and has developed into a leading international Summer Institute in the field of EPP, both experimental and theoretical, providing in addition a very rich outreach programme to teachers and school students. The CORFU2005 Summer Institute on EPP, although based on the general format that has been developed and established in the Corfu Meetings during previous years, is characterized by the fact that it was a full realization of a new idea, which started experimentally in the previous two Corfu Meetings. The successful new ingredient was that three European Marie Curie Research Training Networks decided to hold their Workshops in Corfu during September 2005 and they managed to coordinate the educational part of their meetings to a huge Summer School called `The 8th Hellenic School on Elementary Particle Physics' (4–11 September). The European Networks which joined forces to materialize this project and the corresponding dates of their own Workshops are:The Third Generation as a Probe for New Physics: Experimental and Technological Approach (4–11 September)The Quest for Unification Theory Confronts Experiment (11–18 September)Constituents Fundamental Forces and Symmetries of the Universe (20–26 September)To these Workshops has been added a Satellite one called `Noncommutative Geometry in Field and String Theory', and some extra speakers have been invited to complement the full programme of CORFU2005, some of whom have integrated into the Workshop's programme.The result was indeed very successful! An impressive aspect is that the CORFU2005 had the most massive participation so far attracting around 350 scientists. Among them around 200 young scientists (100 graduate students and 100 post doctoral scientists) and around 150 senior scientists. Therefore, among others, CORFU2005 hosted one of the largest Summer Schools in our field. Internationally leading scientists have been gathered in the CORFU2005 in the various Workshops and the School and have created a stimulating scientific atmosphere for themselves and for the young scientists. The contributions of all speakers can be found inhttp://corfu2005.physics.uoi.gr.Most of them have contributed to the present proceedings, while the contributions of the last week can be found inFortsch. Phys.54, Issue 5–6 (May 2006). In parallel to the main scientific programme a very interesting, rich and successful outreach programme was held in collaboration with the local Department of the Greek Physical Society and the Laboratory of Physical Science in Corfu (EKFE).The success of the CORFU2005 was the best advertisement concerning the long standing efforts to establish the `European Institute of Science and their Applications', which eventually was founded last spring in Corfu. The new Institute hopes to be the permanent extension of the Corfu Summer Institutes on EPP and has an additional aim to upgrade them in the sense that the attracted first class scientists would produce locally a significant research output.We would like to thank everybody very much who contributed to the success of CORFU2005. We would like specifically to thank all speakers and organizers, the conference secretary and the school officer (please consulthttp://corfu2005.physics.uoi.gr) and the group of graduate students who helped in various ways and contributed in a very significant manner in the success of CORFU2005. In addition we would like to thank our sponsors, whose contribution made possible the CORFU2005:European Research Training Network:The Third Generation as a Probe for New Physics: Experimental and Technological Approach; The Quest for Unification Theory Confronts Experiment; Constituents Fundamental Forces and Symmetries of the UniverseGreek Ministry of EducationMunicipality of Corfu and the Municipal Development Enterprise (ANEDK)Latsis InstitutionNational Technical University and Ionian UniversityCERNDESYMax--Planck--Institute for Physics, MunichSommerfeld Center for Theoretical PhysicsNational Center of Scientific Research ``Demokritos'' and the Greek Atomic Energy CommissionOlympic AirlinesThe Companies: Educational Tetras, Infoware, Kleos, Terra KerkyraThe telecommunication companies OTE and FORTHNETKonstantinos Anagnostopoulos, Ignatios Antoniadis, George Fanourakis, Alexandros Kehagias, Aurore Savoy-Navarro, Julius Wess and George ZoupanosEditors
%-------------------------------------------------------------------------
% NewRecord: 31. hep-th/0506062 Anagnostopoulos:2005cy SlacID: 684415 DOI: 10.1088/1126-6708/2005/09/046 Citations:  27 ( 21 ) Pages: 27
\item Konstantinos N. Anagnostopoulos, Takehiro Azuma, Keiichi Nagao, Jun Nishimura, {\it ``Impact of supersymmetry on the nonperturbative dynamics of fuzzy spheres''}, \href{https://www.doi.org/10.1088/1126-6708/2005/09/046}{JHEP {\bf 09} (2005) 046} \href{https://arxiv.org/abs/hep-th/0506062}{[arXiv:hep-th/0506062]}, \href{https://www.doi.org/10.1088/1126-6708/2005/09/046}{[doi:10.1088/1126-6708/2005/09/046]}.
\\\href{https://inspirehep.net/literature/?q=refersto%3Arecid%3A684415}{Cited by 27 (21)} articles

We study a 4d supersymmetric matrix model with a cubic term, which incorporates fuzzy spheres as classical solutions, using Monte Carlo simulations and perturbative calculations. The fuzzy sphere in the supersymmetric model turns out to be always stable if the large-N limit is taken in such a way that various correlation functions scale. This is in striking contrast to analogous bosonic models, where the fuzzy sphere decays into the pure Yang-Mills vacuum due to quantum effects when the coefficient of the cubic term becomes smaller than a critical value. We also find that the power-law tail of the eigenvalue distribution, which exists in the supersymmetric model without the cubic term, disappears in the presence of the fuzzy sphere in the large-N limit. Coincident fuzzy spheres turn out to be unstable, which implies that the dynamically generated gauge group is U(1).
%-------------------------------------------------------------------------
% NewRecord: 32. hep-lat/0402031 Ambjorn:2004jk SlacID: 645128 DOI: 10.1103/PhysRevD.70.035010 Citations:  14 ( 4 ) Pages: 7
\item Jan Ambjorn, Konstantinos N. Anagnostopoulos, Jun Nishimura, Jacobus J.M. Verbaarschot, {\it ``Noncommutativity of the zero chemical potential limit and the thermodynamic limit in finite density systems''}, \href{https://www.doi.org/10.1103/PhysRevD.70.035010}{Phys.Rev.D {\bf 70} (2004) 035010} \href{https://arxiv.org/abs/hep-lat/0402031}{[arXiv:hep-lat/0402031]}, \href{https://www.doi.org/10.1103/PhysRevD.70.035010}{[doi:10.1103/PhysRevD.70.035010]}.
\\\href{https://inspirehep.net/literature/?q=refersto%3Arecid%3A645128}{Cited by 14 (4)} articles

Monte Carlo simulations of finite density systems are often plagued by the complex action problem. We point out that there exists certain non-commutativity in the zero chemical potential limit and the thermodynamic limit when one tries to study such systems by reweighting techniques. This is demonstrated by explicit calculations in a Random Matrix Theory, which is thought to be a simple qualitative model for finite density QCD. The factorization method allows us to understand how the non-commutativity, which appears at the intermediate steps, cancels in the end results for physical observables.
%-------------------------------------------------------------------------
% NewRecord: 33. hep-lat/0310004 Ambjorn:2003rr SlacID: 629664 DOI: 10.1142/9789812702845_0030 Citations:  1 ( 0 ) Pages: 5
\item Jan Ambjorn, Konstantinos N. Anagnostopoulos, Jun Nishimura, Jacobus J.M. Verbaarschot, {\it ``The Factorization method for simulating systems with a complex action''}, \href{https://arxiv.org/abs/hep-lat/0310004}{arXiv:hep-lat/0310004}, \href{https://www.doi.org/10.1142/9789812702845_0030}{[doi:10.1142/9789812702845\_0030]}.
\\\href{https://inspirehep.net/literature/?q=refersto%3Arecid%3A629664}{Cited by 1 (0)} articles

We propose a method for Monte Carlo simulations of systems with a complex action. The method has the advantages of being in principle applicable to any such system and provides a solution to the overlap problem. We apply it in random matrix theory of finite density QCD where we compare with analytic results. In this model we find non--commutativity of the limits muto 0 and Ntoinfty which could be of relevance in QCD at finite density.
%-------------------------------------------------------------------------
% NewRecord: 34. hep-lat/0309076 Ambjorn:2003ms SlacID: 628166 DOI: 10.1016/S0920-5632(03)02631-8 Citations:  0 ( 0 ) Pages: 3
\item J. Ambjorn, K.N. Anagnostopoulos, J. Nishimura, J.J.M. Verbaarschot, {\it ``The Factorization method for Monte Carlo simulations of systems with a complex action''}, \href{https://www.doi.org/10.1016/S0920-5632(03)02631-8}{Nucl.Phys.B Proc.Suppl. {\bf 129} (2004) } \href{https://arxiv.org/abs/hep-lat/0309076}{[arXiv:hep-lat/0309076]}, \href{https://www.doi.org/10.1016/S0920-5632(03)02631-8}{[doi:10.1016/S0920-5632(03)02631-8]}.

We propose a method for Monte Carlo simulations of systems with a complex action. The method has the advantages of being in principle applicable to any such system and provides a solution to the overlap problem. In some cases, like in the IKKT matrix model, a finite size scaling extrapolation can provide results for systems whose size would make it prohibitive to simulate directly.
%-------------------------------------------------------------------------
% NewRecord: 35. hep-lat/0208025 Ambjorn:2002pz SlacID: 593052 DOI: 10.1088/1126-6708/2002/10/062 Citations:  101 ( 73 ) Pages: 27
\item J. Ambjorn, K.N. Anagnostopoulos, J. Nishimura, J.J.M. Verbaarschot, {\it ``The Factorization method for systems with a complex action: A Test in random matrix theory for finite density QCD''}, \href{https://www.doi.org/10.1088/1126-6708/2002/10/062}{JHEP {\bf 10} (2002) 062} \href{https://arxiv.org/abs/hep-lat/0208025}{[arXiv:hep-lat/0208025]}, \href{https://www.doi.org/10.1088/1126-6708/2002/10/062}{[doi:10.1088/1126-6708/2002/10/062]}.
\\\href{https://inspirehep.net/literature/?q=refersto%3Arecid%3A593052}{Cited by 101 (73)} articles

Monte Carlo simulations of systems with a complex action are known to be extremely difficult. A new approach to this problem based on a factorization property of distribution functions of observables has been proposed recently. The method can be applied to any system with a complex action, and it eliminates the so-called overlap problem completely. We test the new approach in a Random Matrix Theory for finite density QCD, where we are able to reproduce the exact results for the quark number density. The achieved system size is large enough to extract the thermodynamic limit. Our results provide a clear understanding of how the expected first order phase transition is induced by the imaginary part of the action.
%-------------------------------------------------------------------------
% NewRecord: 36. hep-lat/0112035 Anagnostopoulos:2001cb SlacID: 568670 DOI: 10.1142/S0129183102003334 Citations:  7 ( 1 ) Pages: 12
\item K.N. Anagnostopoulos, Wolfgang Bietenholz, J. Nishimura, {\it ``The Area law in matrix models for large N QCD strings''}, \href{https://www.doi.org/10.1142/S0129183102003334}{Int.J.Mod.Phys.C {\bf 13} (2002) } \href{https://arxiv.org/abs/hep-lat/0112035}{[arXiv:hep-lat/0112035]}, \href{https://www.doi.org/10.1142/S0129183102003334}{[doi:10.1142/S0129183102003334]}.
\\\href{https://inspirehep.net/literature/?q=refersto%3Arecid%3A568670}{Cited by 7 (1)} articles

We study the question whether matrix models obtained in the zero volume limit of 4d Yang-Mills theories can describe large N QCD strings. The matrix model we use is a variant of the Eguchi-Kawai model in terms of Hermitian matrices, but without any twists or quenching. This model was originally proposed as a toy model of the IIB matrix model. In contrast to common expectations, we do observe the area law for Wilson loops in a significant range of scale of the loop area. Numerical simulations show that this range is stable as N increases up to 768, which strongly suggests that it persists in the large N limit. Hence the equivalence to QCD strings may hold for length scales inside a finite regime.
%-------------------------------------------------------------------------
% NewRecord: 37. hep-lat/0110092 Ambjorn:2001xu SlacID: 564302 DOI: 10.1016/S0920-5632(01)01775-3 Citations:  1 ( 1 ) Pages: 3
\item J. Ambjorn, K.N. Anagnostopoulos, A. Krasnitz, {\it ``Real time dynamics of a hot Yang-Mills theory: A Numerical analysis''}, \href{https://www.doi.org/10.1016/S0920-5632(01)01775-3}{Nucl.Phys.B Proc.Suppl. {\bf 106} (2002) } \href{https://arxiv.org/abs/hep-lat/0110092}{[arXiv:hep-lat/0110092]}, \href{https://www.doi.org/10.1016/S0920-5632(01)01775-3}{[doi:10.1016/S0920-5632(01)01775-3]}.
\\\href{https://inspirehep.net/literature/?q=refersto%3Arecid%3A564302}{Cited by 1 (1)} articles

We discuss recent results obtained from simulations of high temperature, classical, real time dynamics of SU(2) Yang-Mills theory at temperatures of the order of the electroweak scale. Measurements of gauge covariant and gauge invariant autocorrelations of the fields indicate that the ASY-Bodecker scenario is irrelevant at these temperatures.
%-------------------------------------------------------------------------
% NewRecord: 38. hep-lat/0110094 Ambjorn:2001xw SlacID: 564304 DOI: 10.1016/S0920-5632(01)01899-0 Citations:  0 ( 0 ) Pages: 3
\item J. Ambjorn, K.N. Anagnostopoulos, Wolfgang Bietenholz, F. Hofheinz, J. Nishimura, {\it ``On the quantum geometry of string theory''}, \href{https://www.doi.org/10.1016/S0920-5632(01)01899-0}{Nucl.Phys.B Proc.Suppl. {\bf 106} (2002) } \href{https://arxiv.org/abs/hep-lat/0110094}{[arXiv:hep-lat/0110094]}, \href{https://www.doi.org/10.1016/S0920-5632(01)01899-0}{[doi:10.1016/S0920-5632(01)01899-0]}.

The IKKT or IIB matrix model has been proposed as a non-perturbative definition of type IIB superstring theories. It has the attractive feature that space--time appears dynamically. It is possible that lower dimensional universes dominate the theory, therefore providing a dynamical solution to the reduction of space--time dimensionality. We summarize recent works that show the central role of the phase of the fermion determinant in the possible realization of such a scenario.
%-------------------------------------------------------------------------
% NewRecord: 39. hep-ph/0109080 Anagnostopoulos:2001dh SlacID: 562558 DOI: 10.1103/PhysRevD.64.125006 Citations:  42 ( 42 ) Pages: 13
\item K.N. Anagnostopoulos, M. Axenides, E.G. Floratos, N. Tetradis, {\it ``Large gauged Q balls''}, \href{https://www.doi.org/10.1103/PhysRevD.64.125006}{Phys.Rev.D {\bf 64} (2001) 125006} \href{https://arxiv.org/abs/hep-ph/0109080}{[arXiv:hep-ph/0109080]}, \href{https://www.doi.org/10.1103/PhysRevD.64.125006}{[doi:10.1103/PhysRevD.64.125006]}.
\\\href{https://inspirehep.net/literature/?q=refersto%3Arecid%3A562558}{Cited by 42 (42)} articles

We study Q-balls associated with local U(1) symmetries. Such Q-balls are expected to become unstable for large values of their charge because of the repulsion mediated by the gauge force. We consider the possibility that the repulsion is eliminated through the presence in the interior of the Q-ball of fermions with charge opposite to that of the scalar condensate. Another possibility is that two scalar condensates of opposite charge form in the interior. We demonstrate that both these scenaria can lead to the existence of classically stable, large, gauged Q-balls. We present numerical solutions, as well as an analytical treatment of the ``thin-wall'' limit.
%-------------------------------------------------------------------------
% NewRecord: 40. hep-th/0108041 Anagnostopoulos:2001yb SlacID: 561215 DOI: 10.1103/PhysRevD.66.106008 Citations:  121 ( 68 ) Pages: 4
\item K.N. Anagnostopoulos, J. Nishimura, {\it ``New approach to the complex action problem and its application to a nonperturbative study of superstring theory''}, \href{https://www.doi.org/10.1103/PhysRevD.66.106008}{Phys.Rev.D {\bf 66} (2002) 106008} \href{https://arxiv.org/abs/hep-th/0108041}{[arXiv:hep-th/0108041]}, \href{https://www.doi.org/10.1103/PhysRevD.66.106008}{[doi:10.1103/PhysRevD.66.106008]}.
\\\href{https://inspirehep.net/literature/?q=refersto%3Arecid%3A561215}{Cited by 121 (68)} articles

Monte Carlo simulations of a system whose action has an imaginary part are considered to be extremely difficult. We propose a new approach to this `complex-action problem', which utilizes a factorization property of distribution functions. The basic idea is quite general, and it removes the so-called overlap problem completely. Here we apply the method to a nonperturbative study of superstring theory using its matrix formulation. In this particular example, the distribution function turns out to be positive definite, which allows us to reduce the problem even further. Our numerical results suggest an intuitive explanation for the dynamical generation of 4d space-time.
%-------------------------------------------------------------------------
% NewRecord: 41. hep-th/0104260 Ambjorn:2001xs SlacID: 556015 DOI: 10.1103/PhysRevD.65.086001 Citations:  54 ( 16 ) Pages: 15
\item Jan Ambjorn, K.N. Anagnostopoulos, Wolfgang Bietenholz, F. Hofheinz, J. Nishimura, {\it ``On the spontaneous breakdown of Lorentz symmetry in matrix models of superstrings''}, \href{https://www.doi.org/10.1103/PhysRevD.65.086001}{Phys.Rev.D {\bf 65} (2002) 086001} \href{https://arxiv.org/abs/hep-th/0104260}{[arXiv:hep-th/0104260]}, \href{https://www.doi.org/10.1103/PhysRevD.65.086001}{[doi:10.1103/PhysRevD.65.086001]}.
\\\href{https://inspirehep.net/literature/?q=refersto%3Arecid%3A556015}{Cited by 54 (16)} articles

In string or M theories, the spontaneous breaking of 10D or 11D Lorentz symmetry is required to describe our space-time. A direct approach to this issue is provided by the IIB matrix model. We study its 4D version, which corresponds to the zero volume limit of 4D super SU(N) Yang-Mills theory. Based on the moment of inertia as a criterion, spontaneous symmetry breaking (SSB) seems to occur, so that only one extended direction remains, as first observed by Bialas, Burda et al. However, using Wilson loops as probes of space-time we do not observe any sign of SSB in Monte Carlo simulations where N is as large as 48. This agrees with an earlier observation that the phase of the fermionic integral, which is absent in the 4D model, should play a crucial role if SSB of Lorentz symmetry really occurs in the 10D IIB matrix model.
%-------------------------------------------------------------------------
% NewRecord: 42. hep-ph/0101309 Ambjorn:2001ga SlacID: 552548 DOI: 10.1088/1126-6708/2001/06/069 Citations:  6 ( 4 ) Pages: 13
\item Jan Ambjorn, K.N. Anagnostopoulos, A. Krasnitz, {\it ``High temperature, classical, real time dynamics of nonAbelian gauge theories as seen by a computer''}, \href{https://www.doi.org/10.1088/1126-6708/2001/06/069}{JHEP {\bf 06} (2001) 069} \href{https://arxiv.org/abs/hep-ph/0101309}{[arXiv:hep-ph/0101309]}, \href{https://www.doi.org/10.1088/1126-6708/2001/06/069}{[doi:10.1088/1126-6708/2001/06/069]}.
\\\href{https://inspirehep.net/literature/?q=refersto%3Arecid%3A552548}{Cited by 6 (4)} articles

We test at the electroweak scale the recently proposed elaborate theoretical scenario for real-time dynamics of non-abelian gauge theories at high temperature. We see no sign of the predicted behavior. This indicates that perturbative concepts like color conductivity and Landau damping might be irrelevant at temperatures corresponding to the electroweak scale.
%-------------------------------------------------------------------------
% NewRecord: 43. hep-th/0012061 Anagnostopoulos:2000mn SlacID: 538190 DOI: 10.1088/1126-6708/2001/04/024 Citations:  10 ( 6 ) Pages: 20
\item K.N. Anagnostopoulos, J. Nishimura, P. Olesen, {\it ``Noncommutative string world sheets from matrix models''}, \href{https://www.doi.org/10.1088/1126-6708/2001/04/024}{JHEP {\bf 04} (2001) 024} \href{https://arxiv.org/abs/hep-th/0012061}{[arXiv:hep-th/0012061]}, \href{https://www.doi.org/10.1088/1126-6708/2001/04/024}{[doi:10.1088/1126-6708/2001/04/024]}.
\\\href{https://inspirehep.net/literature/?q=refersto%3Arecid%3A538190}{Cited by 10 (6)} articles

We study dynamical effects of introducing noncommutativity on string worldsheets by using a matrix model obtained from the zero-volume limit of four-dimensional SU(N) Yang-Mills theory. Although the dimensionless noncommutativity parameter is of order 1/N, its effect is found to be non-negligible even in the large N limit due to the existence of higher Fourier modes. We find that the Poisson bracket grows much faster than the Moyal bracket as we increase N, which means in particular that the two quantities do not coincide in the large N limit. The well-known instability of bosonic worldsheets is shown to be cured by the noncommutativity, leading to a well-defined bosonic string theory, which may be interpreted as a theory for QCD strings.
%-------------------------------------------------------------------------
% NewRecord: 44. hep-lat/0009030 Ambjorn:2000xj SlacID: 534033 DOI: 10.1016/S0920-5632(01)00877-5 Citations:  9 ( 7 ) Pages: 4
\item Jan Ambjorn, K.N. Anagnostopoulos, Wolfgang Bietenholz, T. Hotta, J. Nishimura, {\it ``Simulating simplified versions of the IKKT matrix model''}, \href{https://www.doi.org/10.1016/S0920-5632(01)00877-5}{Nucl.Phys.B Proc.Suppl. {\bf 94} (2001) } \href{https://arxiv.org/abs/hep-lat/0009030}{[arXiv:hep-lat/0009030]}, \href{https://www.doi.org/10.1016/S0920-5632(01)00877-5}{[doi:10.1016/S0920-5632(01)00877-5]}.
\\\href{https://inspirehep.net/literature/?q=refersto%3Arecid%3A534033}{Cited by 9 (7)} articles

We simulate a supersymmetric matrix model obtained from dimensional reduction of 4d SU(N) super Yang-Mills theory (a 4d counter part of the IKKT model or IIB matrix model). The eigenvalue distribution determines the space structure. The measurement of Wilson loop correlators reveals a universal large N scaling. Eguchi-Kawai equivalence may hold in a finite range of scale, which is also true for the bosonic case. We finally report on simulations of a low energy approximation of the 10d IKKT model, where we omit the phase of the Pfaffian and look for evidence for a spontaneous Lorentz symmetry breaking.
%-------------------------------------------------------------------------
% NewRecord: 45. hep-th/0005147 Ambjorn:2000dx SlacID: 527438 DOI: 10.1088/1126-6708/2000/07/011 Citations:  110 ( 55 ) Pages: 26
\item Jan Ambjorn, K.N. Anagnostopoulos, Wolfgang Bietenholz, T. Hotta, J. Nishimura, {\it ``Monte Carlo studies of the IIB matrix model at large N''}, \href{https://www.doi.org/10.1088/1126-6708/2000/07/011}{JHEP {\bf 07} (2000) 011} \href{https://arxiv.org/abs/hep-th/0005147}{[arXiv:hep-th/0005147]}, \href{https://www.doi.org/10.1088/1126-6708/2000/07/011}{[doi:10.1088/1126-6708/2000/07/011]}.
\\\href{https://inspirehep.net/literature/?q=refersto%3Arecid%3A527438}{Cited by 110 (55)} articles

The low-energy effective theory of the IIB matrix model developed by H. Aoki et al. is written down explicitly in terms of bosonic variables only. The effective theory is then studied by Monte Carlo simulations in order to investigate the possibility of a spontaneous breakdown of Lorentz invariance. The imaginary part of the effective action, which causes the so-called sign problem in the simulation, is dropped by hand. The extent of the eigenvalue distribution of the bosonic matrices shows a power-law large N behavior, consistent with a simple branched-polymer prediction. We observe, however, that the eigenvalue distribution becomes more and more isotropic in the ten-dimensional space-time as we increase N. This suggests that if the spontaneous breakdown of Lorentz invariance really occurs in the IIB matrix model, a crucial role must be played by the imaginary part of the effective action.
%-------------------------------------------------------------------------
% NewRecord: 46. hep-th/0101084 Ambjorn:2000dj SlacID: 539882 DOI: NoDOI Citations:  7 ( 5 ) Pages: 6
\item Jan Ambjorn, K.N. Anagnostopoulos, Wolfgang Bietenholz, T. Hotta, J. Nishimura, {\it ``Monte Carlo studies of the dimensionally reduced 4-D SU(N) superYang-Mills theory''}, \href{https://arxiv.org/abs/hep-th/0101084}{arXiv:hep-th/0101084}.
\\\href{https://inspirehep.net/literature/?q=refersto%3Arecid%3A539882}{Cited by 7 (5)} articles

We simulate a supersymmetric matrix model obtained from dimensional reduction of 4d SU(N) super Yang-Mills theory. The model is well defined for finite N and it is found that the large N limit obtained by keeping g2 N fixed gives rise to well defined operators which represent string amplitudes. The space-time structure which arises dynamically from the eigenvalues of the bosonic matrices is discussed, as well as the effect of supersymmetry on the dynamical properties of the model. Eguchi-Kawai equivalence of this model to ordinary gauge theory does hold within a finite range of scale. We report on new simulations of the bosonic model for N up to 768 that confirm this property, which comes as a surprise since no quenching or twist is introduced.
%-------------------------------------------------------------------------
% NewRecord: 47. hep-th/0003208 Ambjorn:2000bf SlacID: 525317 DOI: 10.1088/1126-6708/2000/07/013 Citations:  123 ( 64 ) Pages: 36
\item Jan Ambjorn, K.N. Anagnostopoulos, Wolfgang Bietenholz, T. Hotta, J. Nishimura, {\it ``Large N dynamics of dimensionally reduced 4-D SU(N) superYang-Mills theory''}, \href{https://www.doi.org/10.1088/1126-6708/2000/07/013}{JHEP {\bf 07} (2000) 013} \href{https://arxiv.org/abs/hep-th/0003208}{[arXiv:hep-th/0003208]}, \href{https://www.doi.org/10.1088/1126-6708/2000/07/013}{[doi:10.1088/1126-6708/2000/07/013]}.
\\\href{https://inspirehep.net/literature/?q=refersto%3Arecid%3A525317}{Cited by 123 (64)} articles

We perform Monte Carlo simulations of a supersymmetric matrix model, which is obtained by dimensional reduction of 4D SU(N) super Yang-Mills theory. The model can be considered as a four-dimensional counterpart of the IIB matrix model. We extract the space-time structure represented by the eigenvalues of bosonic matrices. In particular we compare the large N behavior of the space-time extent with the result obtained from a low energy effective theory. We measure various Wilson loop correlators which represent string amplitudes and we observe a nontrivial universal scaling in N. We also observe that the Eguchi-Kawai equivalence to ordinary gauge theory does hold at least within a finite range of scale. Comparison with the results for the bosonic case clarifies the role of supersymmetry in the large N dynamics. It does affect the multi-point correlators qualitatively, but the Eguchi-Kawai equivalence is observed even in the bosonic case.
%-------------------------------------------------------------------------
% NewRecord: 48. hep-th/9910232 Loll:1999uu SlacID: 509292 DOI: 10.1016/S0920-5632(00)00776-3 Citations:  7 ( 5 ) Pages: 4
\item R. Loll, Jan Ambjorn, K.N. Anagnostopoulos, {\it ``Making the gravitational path integral more Lorentzian, or life beyond Liouville gravity''}, \href{https://www.doi.org/10.1016/S0920-5632(00)00776-3}{Nucl.Phys.B Proc.Suppl. {\bf 88} (2000) } \href{https://arxiv.org/abs/hep-th/9910232}{[arXiv:hep-th/9910232]}, \href{https://www.doi.org/10.1016/S0920-5632(00)00776-3}{[doi:10.1016/S0920-5632(00)00776-3]}.
\\\href{https://inspirehep.net/literature/?q=refersto%3Arecid%3A509292}{Cited by 7 (5)} articles

In two space-time dimensions, there is a theory of Lorentzian quantum gravity which can be defined by a rigorous, non-perturbative path integral and is inequivalent to the well-known theory of (Euclidean) quantum Liouville gravity. It has a number of appealing features: i) its quantum geometry is non-fractal, ii) it remains consistent when coupled to matter, even beyond the c=1 barrier, iii) it is closer to canonical quantization approaches than previous path-integral formulations, and iv) its construction generalizes to higher dimensions.
%-------------------------------------------------------------------------
% NewRecord: 49. hep-lat/9909129 Ambjorn:1999yv SlacID: 507367 DOI: 10.1103/PhysRevD.61.044010 Citations:  49 ( 17 ) Pages: 22
\item Jan Ambjorn, K.N. Anagnostopoulos, R. Loll, {\it ``Crossing the c = 1 barrier in 2-D Lorentzian quantum gravity''}, \href{https://www.doi.org/10.1103/PhysRevD.61.044010}{Phys.Rev.D {\bf 61} (2000) 044010} \href{https://arxiv.org/abs/hep-lat/9909129}{[arXiv:hep-lat/9909129]}, \href{https://www.doi.org/10.1103/PhysRevD.61.044010}{[doi:10.1103/PhysRevD.61.044010]}.
\\\href{https://inspirehep.net/literature/?q=refersto%3Arecid%3A507367}{Cited by 49 (17)} articles

In an extension of earlier work we investigate the behaviour of two-dimensional Lorentzian quantum gravity under coupling to a conformal field theory with c>1. This is done by analyzing numerically a system of eight Ising models (corresponding to c=4) coupled to dynamically triangulated Lorentzian geometries. It is known that a single Ising model couples weakly to Lorentzian quantum gravity, in the sense that the Hausdorff dimension of the ensemble of two-geometries is two (as in pure Lorentzian quantum gravity) and the matter behaviour is governed by the Onsager exponents. By increasing the amount of matter to 8 Ising models, we find that the geometry of the combined system has undergone a phase transition. The new phase is characterized by an anomalous scaling of spatial length relative to proper time at large distances, and as a consequence the Hausdorff dimension is now three. In spite of this qualitative change in the geometric sector, and a very strong interaction between matter and geometry, the critical exponents of the Ising model retain their Onsager values. This provides evidence for the conjecture that the KPZ values of the critical exponents in 2d Euclidean quantum gravity are entirely due to the presence of baby universes. Lastly, we summarize the lessons learned so far from 2d Lorentzian quantum gravity.
%-------------------------------------------------------------------------
% NewRecord: 50. hep-lat/9907027 Ambjorn:1999ix SlacID: 504781 DOI: 10.1088/1126-6708/1999/08/016 Citations:  37 ( 25 ) Pages: 20
\item Jan Ambjorn, K.N. Anagnostopoulos, J. Jurkiewicz, {\it ``Abelian gauge fields coupled to simplicial quantum gravity''}, \href{https://www.doi.org/10.1088/1126-6708/1999/08/016}{JHEP {\bf 08} (1999) 016} \href{https://arxiv.org/abs/hep-lat/9907027}{[arXiv:hep-lat/9907027]}, \href{https://www.doi.org/10.1088/1126-6708/1999/08/016}{[doi:10.1088/1126-6708/1999/08/016]}.
\\\href{https://inspirehep.net/literature/?q=refersto%3Arecid%3A504781}{Cited by 37 (25)} articles

We study the coupling of Abelian gauge theories to four-dimensional simplicial quantum gravity. The gauge fields live on dual links. This is the correct formulation if we want to compare the effect of gauge fields on geometry with similar effects studied so far for scalar fields. It shows that gauge fields couple equally weakly to geometry as scalar fields, and it offers an understanding of the relation between measure factors and Abelian gauge fields observed so-far.
%-------------------------------------------------------------------------
% NewRecord: 51. hep-lat/9908054 Ambjorn:1999mc SlacID: 506246 DOI: 10.1016/S0920-5632(00)91790-0 Citations:  8 ( 3 ) Pages: 3
\item Jan Ambjorn, K.N. Anagnostopoulos, R. Loll, {\it ``On the phase diagram of 2-d Lorentzian quantum gravity''}, \href{https://www.doi.org/10.1016/S0920-5632(00)91790-0}{Nucl.Phys.B Proc.Suppl. {\bf 83} (2000) } \href{https://arxiv.org/abs/hep-lat/9908054}{[arXiv:hep-lat/9908054]}, \href{https://www.doi.org/10.1016/S0920-5632(00)91790-0}{[doi:10.1016/S0920-5632(00)91790-0]}.
\\\href{https://inspirehep.net/literature/?q=refersto%3Arecid%3A506246}{Cited by 8 (3)} articles

The phase diagram of 2d Lorentzian quantum gravity (LQG) coupled to conformal matter is studied. A phase transition is observed at c=c{rm crit} (1/2<c{rm crit}<4) which can be thought of as the analogue of the c=1 barrier of Euclidean quantum gravity (EQG). The non--trivial properties of the quantum geometry are discussed.
%-------------------------------------------------------------------------
% NewRecord: 52. hep-th/9904012 Ambjorn:1999gi SlacID: 497759 DOI: 10.1103/PhysRevD.60.104035 Citations:  66 ( 27 ) Pages: 24
\item Jan Ambjorn, K.N. Anagnostopoulos, R. Loll, {\it ``A New perspective on matter coupling in 2-D quantum gravity''}, \href{https://www.doi.org/10.1103/PhysRevD.60.104035}{Phys.Rev.D {\bf 60} (1999) 104035} \href{https://arxiv.org/abs/hep-th/9904012}{[arXiv:hep-th/9904012]}, \href{https://www.doi.org/10.1103/PhysRevD.60.104035}{[doi:10.1103/PhysRevD.60.104035]}.
\\\href{https://inspirehep.net/literature/?q=refersto%3Arecid%3A497759}{Cited by 66 (27)} articles

We provide compelling evidence that a previously introduced model of non-perturbative 2d Lorentzian quantum gravity exhibits (two-dimensional) flat-space behaviour when coupled to Ising spins. The evidence comes from both a high-temperature expansion and from Monte Carlo simulations of the combined gravity-matter system. This weak-coupling behaviour lends further support to the conclusion that the Lorentzian model is a genuine alternative to Liouville quantum gravity in two dimensions, with a different, and much `smoother' critical behaviour.
%-------------------------------------------------------------------------
% NewRecord: 53. hep-lat/9808027 Ambjorn:1998pz SlacID: 475119 DOI: 10.1088/1126-6708/1998/11/022 Citations:  22 ( 16 ) Pages: 15
\item Jan Ambjorn, K.N. Anagnostopoulos, T. Ichihara, L. Jensen, Y. Watabiki, {\it ``Quantum geometry and diffusion''}, \href{https://www.doi.org/10.1088/1126-6708/1998/11/022}{JHEP {\bf 11} (1998) 022} \href{https://arxiv.org/abs/hep-lat/9808027}{[arXiv:hep-lat/9808027]}, \href{https://www.doi.org/10.1088/1126-6708/1998/11/022}{[doi:10.1088/1126-6708/1998/11/022]}.
\\\href{https://inspirehep.net/literature/?q=refersto%3Arecid%3A475119}{Cited by 22 (16)} articles

We study the diffusion equation in two-dimensional quantum gravity, and show that the spectral dimension is two despite the fact that the intrinsic Hausdorff dimension of the ensemble of two-dimensional geometries is very different from two. We determine the scaling properties of the quantum gravity averaged diffusion kernel.
%-------------------------------------------------------------------------
% NewRecord: 54. hep-lat/9809012 Anagnostopoulos:1998xe SlacID: 475766 DOI: 10.1016/S0920-5632(99)85203-7 Citations:  0 ( 0 ) Pages: 3
\item K.N. Anagnostopoulos, {\it ``Scaling and quantum geometry in 2-D gravity''}, \href{https://www.doi.org/10.1016/S0920-5632(99)85203-7}{Nucl.Phys.B Proc.Suppl. {\bf 73} (1999) } \href{https://arxiv.org/abs/hep-lat/9809012}{[arXiv:hep-lat/9809012]}, \href{https://www.doi.org/10.1016/S0920-5632(99)85203-7}{[doi:10.1016/S0920-5632(99)85203-7]}.

We review the status of understanding of the fractal structure of the quantum spacetime of 2d gravity coupled to conformal matter with c <= 1, with emphasis put on the results obtained last year.
%-------------------------------------------------------------------------
% NewRecord: 55. cond-mat/9804137 Anagnostopoulos:1998fd SlacID: 470032 DOI: 10.1023/A:1004583901498 Citations:  13 ( 9 ) Pages: 23
\item K.N. Anagnostopoulos, P. Bialas, G. Thorleifsson, {\it ``The Ising model on a quenched ensemble of c = -5 gravity graphs''}, \href{https://www.doi.org/10.1023/A:1004583901498}{J.Statist.Phys. {\bf 94} (1999) } \href{https://arxiv.org/abs/cond-mat/9804137}{[arXiv:cond-mat/9804137]}, \href{https://www.doi.org/10.1023/A:1004583901498}{[doi:10.1023/A:1004583901498]}.
\\\href{https://inspirehep.net/literature/?q=refersto%3Arecid%3A470032}{Cited by 13 (9)} articles

We study with Monte Carlo methods an ensemble of c=-5 gravity graphs, generated by coupling a conformal field theory with central charge c=-5 to two-dimensional quantum gravity. We measure the fractal properties of the ensemble, such as the string susceptibility exponent gammas and the intrinsic fractal dimensions dH. We find gammas = -1.5(1) and dH = 3.36(4), in reasonable agreement with theoretical predictions. In addition, we study the critical behavior of an Ising model on a quenched ensemble of the c=-5 graphs and show that it agrees, within numerical accuracy, with theoretical predictions for the critical behavior of an Ising model coupled dynamically to two-dimensional quantum gravity, provided the total central charge of the matter sector is c=-5. From this we conjecture that the critical behavior of the Ising model is determined solely by the average fractal properties of the graphs, the coupling to the geometry not playing an important role.
%-------------------------------------------------------------------------
% NewRecord: 56. hep-th/9802020 Ambjorn:1998zs SlacID: 466869 DOI: 10.1088/1126-6708/1998/04/016 Citations:  12 ( 3 ) Pages: 14
\item Jan Ambjorn, K.N. Anagnostopoulos, J. Jurkiewicz, C.F. Kristjansen, {\it ``The Concept of time in 2-D gravity''}, \href{https://www.doi.org/10.1088/1126-6708/1998/04/016}{JHEP {\bf 04} (1998) 016} \href{https://arxiv.org/abs/hep-th/9802020}{[arXiv:hep-th/9802020]}, \href{https://www.doi.org/10.1088/1126-6708/1998/04/016}{[doi:10.1088/1126-6708/1998/04/016]}.
\\\href{https://inspirehep.net/literature/?q=refersto%3Arecid%3A466869}{Cited by 12 (3)} articles

We show that the ``time'' ts defined via spin clusters in the Ising model coupled to 2d gravity leads to a fractal dimension dh(s) = 6 of space-time at the critical point, as advocated by Ishibashi and Kawai. In the unmagnetized phase, however, this definition of Hausdorff dimension breaks down. Numerical measurements are consistent with these results. The same definition leads to dh(s)=16 at the critical point when applied to flat space. The fractal dimension dh(s) is in disagreement with both analytical prediction and numerical determination of the fractal dimension dh(g), which is based on the use of the geodesic distance tg as ``proper time''. There seems to be no simple relation of the kind ts = tg{dh(g)/dh(s)}, as expected by dimensional reasons.
%-------------------------------------------------------------------------
% NewRecord: 57. hep-lat/9709025 Ambjorn:1997nf SlacID: 448238 DOI: 10.1016/S0920-5632(97)00890-6 Citations:  6 ( 6 ) Pages: 3
\item Jan Ambjorn, K.N. Anagnostopoulos, G. Thorleifsson, {\it ``The quantum space-time of c > 0 2-D gravity''}, \href{https://www.doi.org/10.1016/S0920-5632(97)00890-6}{Nucl.Phys.B Proc.Suppl. {\bf 63} (1998) } \href{https://arxiv.org/abs/hep-lat/9709025}{[arXiv:hep-lat/9709025]}, \href{https://www.doi.org/10.1016/S0920-5632(97)00890-6}{[doi:10.1016/S0920-5632(97)00890-6]}.
\\\href{https://inspirehep.net/literature/?q=refersto%3Arecid%3A448238}{Cited by 6 (6)} articles

We review recent developments in the understanding of the fractal properties of quantum spacetime of 2d gravity coupled to c>0 conformal matter. In particular we discuss bounds put by numerical simulations using dynamical triangulations on the value of the Hausdorff dimension dH obtained from scaling properties of two point functions defined in terms of geodesic distance. Further insight to the fractal structure of spacetime is obtained from the study of the loop length distribution function which reveals that the 0<c<= 1 system has similar geometric properties with pure gravity, whereas the branched polymer structure becomes clear for c >= 5.
%-------------------------------------------------------------------------
% NewRecord: 58. hep-lat/9708014 Ambjorn:1997ap SlacID: 447442 DOI: 10.1016/S0920-5632(97)00893-1 Citations:  5 ( 5 ) Pages: 3
\item Jan Ambjorn, K.N. Anagnostopoulos, Ulrika Magnea, {\it ``Complex zeros of the 2-D Ising model on dynamical random lattices''}, \href{https://www.doi.org/10.1016/S0920-5632(97)00893-1}{Nucl.Phys.B Proc.Suppl. {\bf 63} (1998) } \href{https://arxiv.org/abs/hep-lat/9708014}{[arXiv:hep-lat/9708014]}, \href{https://www.doi.org/10.1016/S0920-5632(97)00893-1}{[doi:10.1016/S0920-5632(97)00893-1]}.
\\\href{https://inspirehep.net/literature/?q=refersto%3Arecid%3A447442}{Cited by 5 (5)} articles

We study the zeros in the complex plane of the partition function for the Ising model coupled to 2d quantum gravity for complex magnetic field and for complex temperature. We compute the zeros by using the exact solution coming from a two matrix model and by Monte Carlo simulations of Ising spins on dynamical triangulations. We present evidence that the zeros form simple one-dimensional patterns in the complex plane, and that the critical behaviour of the system is governed by the scaling of the distribution of singularities near the critical point.
%-------------------------------------------------------------------------
% NewRecord: 59. hep-lat/9709063 Ambjorn:1997vf SlacID: 448560 DOI: 10.1016/S0920-5632(97)00892-X Citations:  3 ( 1 ) Pages: 3
\item Jan Ambjorn, K.N. Anagnostopoulos, T. Ichihara, L. Jensen, N. Kawamoto, Y. Watabiki, K. Yotsuji, {\it ``Intrinsic geometric structure of c = -2 quantum gravity''}, \href{https://www.doi.org/10.1016/S0920-5632(97)00892-X}{Nucl.Phys.B Proc.Suppl. {\bf 63} (1998) } \href{https://arxiv.org/abs/hep-lat/9709063}{[arXiv:hep-lat/9709063]}, \href{https://www.doi.org/10.1016/S0920-5632(97)00892-X}{[doi:10.1016/S0920-5632(97)00892-X]}.
\\\href{https://inspirehep.net/literature/?q=refersto%3Arecid%3A448560}{Cited by 3 (1)} articles

We couple c=-2 matter to 2-dimensional gravity within the framework of dynamical triangulations. We use a very fast algorithm, special to the c=-2 case, in order to test scaling of correlation functions defined in terms of geodesic distance and we determine the fractal dimension dH with high accuracy. We find dH=3.58(4), consistent with a prediction coming from the study of diffusion in the context of Liouville theory, and that the quantum space-time possesses the same fractal properties at all distance scales similarly to the case of pure gravity.
%-------------------------------------------------------------------------
% NewRecord: 60. hep-lat/9706009 Ambjorn:1997sy SlacID: 444077 DOI: 10.1016/S0550-3213(97)00659-7 Citations:  37 ( 11 ) Pages: 46
\item Jan Ambjorn, K. Anagnostopoulos, T. Ichihara, L. Jensen, N. Kawamoto, Y. Watabiki, K. Yotsuji, {\it ``The Quantum space-time of c = -2 gravity''}, \href{https://www.doi.org/10.1016/S0550-3213(97)00659-7}{Nucl.Phys.B {\bf 511} (1998) } \href{https://arxiv.org/abs/hep-lat/9706009}{[arXiv:hep-lat/9706009]}, \href{https://www.doi.org/10.1016/S0550-3213(97)00659-7}{[doi:10.1016/S0550-3213(97)00659-7]}.
\\\href{https://inspirehep.net/literature/?q=refersto%3Arecid%3A444077}{Cited by 37 (11)} articles

We study the fractal structure of space-time of two-dimensional quantum gravity coupled to c=-2 conformal matter by means of computer simulations. We find that the intrinsic Hausdorff dimension dH = 3.58 +/- 0.04. This result supports the conjecture dH = -2 alpha1/alpha{-1}, where alphan is the gravitational dressing exponent of a spinless primary field of conformal weight (n+1,n+1), and it disfavours the alternative prediction dH = 2/|gamma|. On the other hand <ln> ~ r{2n} for n>1 with good accuracy, i.e. the the boundary length l has an anomalous dimension relative to the area of the surface.
%-------------------------------------------------------------------------
% NewRecord: 61. hep-lat/9705004 Ambjorn:1997dw SlacID: 442748 DOI: 10.1142/S0217732397001643 Citations:  11 ( 10 ) Pages: 22
\item Jan Ambjorn, K.N. Anagnostopoulos, Ulrika Magnea, {\it ``Singularities of the partition function for the Ising model coupled to 2-D quantum gravity''}, \href{https://www.doi.org/10.1142/S0217732397001643}{Mod.Phys.Lett.A {\bf 12} (1997) } \href{https://arxiv.org/abs/hep-lat/9705004}{[arXiv:hep-lat/9705004]}, \href{https://www.doi.org/10.1142/S0217732397001643}{[doi:10.1142/S0217732397001643]}.
\\\href{https://inspirehep.net/literature/?q=refersto%3Arecid%3A442748}{Cited by 11 (10)} articles

We study the zeros in the complex plane of the partition function for the Ising model coupled to 2d quantum gravity for complex magnetic field and real temperature, and for complex temperature and real magnetic field, respectively. We compute the zeros by using the exact solution coming from a two matrix model and by Monte Carlo simulations of Ising spins on dynamical triangulations. We present evidence that the zeros form simple one-dimensional curves in the complex plane, and that the critical behaviour of the system is governed by the scaling of the distribution of the singularities near the critical point. Despite the small size of the systems studied, we can obtain a reasonable estimate of the (known) critical exponents.
%-------------------------------------------------------------------------
% NewRecord: 62. NoArXiv Bowick:1997zz SlacID: 946529 DOI: NoDOI Citations:  0 ( 0 ) Pages: 5
\item M.J. Bowick, S.M. Catterall, M. Falcioni, G. Thorleifsson, K.N. Anagnostopoulos, {\it ``Simulating crystalline membranes''}.


%-------------------------------------------------------------------------
% NewRecord: 63. hep-lat/9701006 Ambjorn:1996wc SlacID: 428024 DOI: 10.1016/S0550-3213(97)00259-9 Citations:  60 ( 27 ) Pages: 38
\item Jan Ambjorn, K.N. Anagnostopoulos, {\it ``Quantum geometry of 2-D gravity coupled to unitary matter''}, \href{https://www.doi.org/10.1016/S0550-3213(97)00259-9}{Nucl.Phys.B {\bf 497} (1997) } \href{https://arxiv.org/abs/hep-lat/9701006}{[arXiv:hep-lat/9701006]}, \href{https://www.doi.org/10.1016/S0550-3213(97)00259-9}{[doi:10.1016/S0550-3213(97)00259-9]}.
\\\href{https://inspirehep.net/literature/?q=refersto%3Arecid%3A428024}{Cited by 60 (27)} articles

We show that there exists a divergent correlation length in 2d quantum gravity for the matter fields close to the critical point provided one uses the invariant geodesic distance as the measure of distance. The corresponding reparameterization invariant two-point functions satisfy all scaling relations known from the ordinary theory of critical phenomena and the KPZ exponents are determined by the power-like fall off of these two-point functions. The only difference compared to flat space is the appearance of a dynamically generated fractal dimension dh in the scaling relations. We analyze numerically the fractal properties of space-time for Ising and three-states Potts model coupled to 2d dimensional quantum gravity using finite size scaling as well as small distance scaling of invariant correlation functions. Our data are consistent with dh=4, but we cannot rule out completely the conjecture dH = -2alpha1/alpha{-1}, where alpha{-n} is the gravitational dressing exponent of a spin-less primary field of conformal weight (n+1,n+1). We compute the moments <L~n> and the loop-length distribution function and show that the fractal properties associated with these observables are identical, with good accuracy, to the pure gravity case.
%-------------------------------------------------------------------------
% NewRecord: 64. hep-lat/9611032 Ambjorn:1996kb SlacID: 426594 DOI: 10.1016/S0370-2693(97)00183-4 Citations:  29 ( 6 ) Pages: 12
\item Jan Ambjorn, K.N. Anagnostopoulos, T. Ichihara, L. Jensen, N. Kawamoto, Y. Watabiki, K. Yotsuji, {\it ``Quantum geometry of topological gravity''}, \href{https://www.doi.org/10.1016/S0370-2693(97)00183-4}{Phys.Lett.B {\bf 397} (1997) } \href{https://arxiv.org/abs/hep-lat/9611032}{[arXiv:hep-lat/9611032]}, \href{https://www.doi.org/10.1016/S0370-2693(97)00183-4}{[doi:10.1016/S0370-2693(97)00183-4]}.
\\\href{https://inspirehep.net/literature/?q=refersto%3Arecid%3A426594}{Cited by 29 (6)} articles

We study a c=-2 conformal field theory coupled to two-dimensional quantum gravity by means of dynamical triangulations. We define the geodesic distance r on the triangulated surface with N triangles, and show that dim[r~{dH}]= dim[N], where the fractal dimension dH = 3.58 +/- 0.04. This result lends support to the conjecture dH = -2alpha1/alpha{-1}, where alpha{-n} is the gravitational dressing exponent of a spin-less primary field of conformal weight (n+1,n+1), and it disfavors the alternative prediction dH = -2/gamma{str}. On the other hand, we find dim[l] = dim[r~2] with good accuracy, where l is the length of one of the boundaries of a circle with (geodesic) radius r, i.e. the length l has an anomalous dimension relative to the area of the surface. It is further shown that the spectral dimension ds = 1.980 +/- 0.014 for the ensemble of (triangulated) manifolds used. The results are derived using finite size scaling and a very efficient recursive sampling technique known previously to work well for c=-2.
%-------------------------------------------------------------------------
% NewRecord: 65. hep-lat/9608044 Bowick:1996gh SlacID: 421798 DOI: 10.1016/S0920-5632(96)00771-2 Citations:  2 ( 2 ) Pages: 7
\item Mark J. Bowick, S.M. Catterall, M. Falcioni, G. Thorleifsson, K. Anagnostopoulos, {\it ``The Flat phase of fixed connectivity membranes''}, \href{https://www.doi.org/10.1016/S0920-5632(96)00771-2}{Nucl.Phys.B Proc.Suppl. {\bf 53} (1997) } \href{https://arxiv.org/abs/hep-lat/9608044}{[arXiv:hep-lat/9608044]}, \href{https://www.doi.org/10.1016/S0920-5632(96)00771-2}{[doi:10.1016/S0920-5632(96)00771-2]}.
\\\href{https://inspirehep.net/literature/?q=refersto%3Arecid%3A421798}{Cited by 2 (2)} articles

The statistical mechanics of flexible two-dimensional surfaces (membranes) appears in a wide variety of physical settings. In this talk we discuss the simplest case of fixed-connectivity surfaces. We first review the current theoretical understanding of the remarkable flat phase of such membranes. We then summarize the results of a recent large scale Monte Carlo simulation of the simplest conceivable discrete realization of this system cite{BCFTA}. We verify the existence of long-range order, determine the associated critical exponents of the flat phase and compare the results to the predictions of various theoretical models.
%-------------------------------------------------------------------------
% NewRecord: 66. hep-lat/9606012 Ambjorn:1996mk SlacID: 419952 DOI: 10.1016/S0370-2693(96)01222-1 Citations:  36 ( 13 ) Pages: 12
\item Jan Ambjorn, K.N. Anagnostopoulos, Ulrika Magnea, G. Thorleifsson, {\it ``Geometrical interpretation of the KPZ exponents''}, \href{https://www.doi.org/10.1016/S0370-2693(96)01222-1}{Phys.Lett.B {\bf 388} (1996) } \href{https://arxiv.org/abs/hep-lat/9606012}{[arXiv:hep-lat/9606012]}, \href{https://www.doi.org/10.1016/S0370-2693(96)01222-1}{[doi:10.1016/S0370-2693(96)01222-1]}.
\\\href{https://inspirehep.net/literature/?q=refersto%3Arecid%3A419952}{Cited by 36 (13)} articles

We provide evidence that the KPZ exponents in two-dimensional quantum gravity can be interpreted as scaling exponents of correlation functions which are functions of the invariant geodesic distance between the fields.
%-------------------------------------------------------------------------
% NewRecord: 67. hep-lat/9608022 Ambjorn:1996bi SlacID: 421704 DOI: 10.1016/S0920-5632(96)00765-7 Citations:  2 ( 1 ) Pages: 3
\item Jan Ambjorn, K.N. Anagnostopoulos, Ulrika Magnea, G. Thorleifsson, {\it ``Spin spin correlation functions of spin systems coupled to 2-d quantum gravity for 0 < c < 1''}, \href{https://www.doi.org/10.1016/S0920-5632(96)00765-7}{Nucl.Phys.B Proc.Suppl. {\bf 53} (1997) } \href{https://arxiv.org/abs/hep-lat/9608022}{[arXiv:hep-lat/9608022]}, \href{https://www.doi.org/10.1016/S0920-5632(96)00765-7}{[doi:10.1016/S0920-5632(96)00765-7]}.
\\\href{https://inspirehep.net/literature/?q=refersto%3Arecid%3A421704}{Cited by 2 (1)} articles

We perform Monte Carlo simulations of 2-d dynamically triangulated surfaces coupled to Ising and three--states Potts model matter. By measuring spin-spin correlation functions as a function of the geodesic distance we provide substantial evidence for a diverging correlation length at betac. The corresponding scaling exponents are directly related to the KPZ exponents of the matter fields as conjectured in [4] (NPB454(1995)313).
%-------------------------------------------------------------------------
% NewRecord: 68. cond-mat/9603157 Bowick:1996wz SlacID: 417130 DOI: 10.1051/jp1:1996139 Citations:  9 ( 6 ) Pages: 31
\item Mark J. Bowick, Simon M. Catterall, Marco Falcioni, Gudmar Thorleifsson, Konstantinos N. Anagnostopoulos, {\it ``The Flat phase of crystalline membranes''}, \href{https://www.doi.org/10.1051/jp1:1996139}{J.Phys.I(France) {\bf 6} (1996) } \href{https://arxiv.org/abs/cond-mat/9603157}{[arXiv:cond-mat/9603157]}, \href{https://www.doi.org/10.1051/jp1:1996139}{[doi:10.1051/jp1:1996139]}.
\\\href{https://inspirehep.net/literature/?q=refersto%3Arecid%3A417130}{Cited by 9 (6)} articles

We present the results of a high-statistics Monte Carlo simulation of a phantom crystalline (fixed-connectivity) membrane with free boundary. We verify the existence of a flat phase by examining lattices of size up to 128~2. The Hamiltonian of the model is the sum of a simple spring pair potential, with no hard-core repulsion, and bending energy. The only free parameter is the the bending rigidity kappa. In-plane elastic constants are not explicitly introduced. We obtain the remarkable result that this simple model dynamically generates the elastic constants required to stabilise the flat phase. We present measurements of the size (Flory) exponent nu and the roughness exponent zeta. We also determine the critical exponents eta and etau describing the scale dependence of the bending rigidity (kappa(q) sim q~{-eta}) and the induced elastic constants (lambda(q) sim mu(q) sim q~{etau}).
%-------------------------------------------------------------------------
% NewRecord: 69. NoArXiv Bowick:1996zz SlacID: 946531 DOI: NoDOI Citations:  0 ( 0 ) Pages: 9
\item M.J. Bowick, S.M. Catterall, M. Falcioni, G. Thorleifsson, K.N. Anagnostopoulos, {\it ``The elastic properties of a flat crystalline membrane''}.


%-------------------------------------------------------------------------
% NewRecord: 70. hep-lat/9509074 Anagnostopoulos:1995vf SlacID: 399718 DOI: 10.1016/0920-5632(96)00187-9 Citations:  2 ( 2 ) Pages: 9
\item K.N. Anagnostopoulos, Mark J. Bowick, S.M. Catterall, M. Falcioni, G. Thorleifsson, {\it ``The Phase diagram of crystalline surfaces''}, \href{https://arxiv.org/abs/hep-lat/9509074}{arXiv:hep-lat/9509074}, \href{https://www.doi.org/10.1016/0920-5632(96)00187-9}{[doi:10.1016/0920-5632(96)00187-9]}.
\\\href{https://inspirehep.net/literature/?q=refersto%3Arecid%3A399718}{Cited by 2 (2)} articles

We report the status of a high-statistics Monte Carlo simulation of non-self-avoiding crystalline surfaces with extrinsic curvature on lattices of size up to 128~2 nodes. We impose free boundary conditions. The free energy is a gaussian spring tethering potential together with a normal-normal bending energy. Particular emphasis is given to the behavior of the model in the cold phase where we measure the decay of the normal-normal correlation function.
%-------------------------------------------------------------------------
% NewRecord: 71. hep-th/9308091 Anagnostopoulos:1993ep SlacID: 357395 DOI: 10.1016/0370-2693(93)91577-A Citations:  14 ( 12 ) Pages: 13
\item Konstantinos Anagnostopoulos, Mark J. Bowick, Paul Coddington, Marco Falcioni, Le-ping Han, Geoffrey R. Harris, Enzo Marinari, {\it ``Fluid random surfaces with extrinsic curvature. 2.''}, \href{https://www.doi.org/10.1016/0370-2693(93)91577-A}{Phys.Lett.B {\bf 317} (1993) } \href{https://arxiv.org/abs/hep-th/9308091}{[arXiv:hep-th/9308091]}, \href{https://www.doi.org/10.1016/0370-2693(93)91577-A}{[doi:10.1016/0370-2693(93)91577-A]}.
\\\href{https://inspirehep.net/literature/?q=refersto%3Arecid%3A357395}{Cited by 14 (12)} articles

We present the results of an extension of our previous work on large-scale simulations of dynamically triangulated toroidal random surfaces embedded in R~3 with extrinsic curvature. We find that the extrinsic-curvature specific heat peak ceases to grow on lattices with more than 576 nodes and that the location of the peak lamc also stabilizes. The evidence for a true crumpling transition is still weak. If we assume it exists we can say that the finite-size scaling exponent frac {alpha} {nu d} is very close to zero or negative. On the other hand our new data does rule out the observed peak as being a finite-size artifact of the persistence length becoming comparable to the extent of the lattice.
%-------------------------------------------------------------------------
% NewRecord: 72. NoArXiv Anagnostopoulos:1993sv SlacID: 364996 DOI: NoDOI Citations:  0 ( 0 ) Pages: 111
\item Konstantinos Anagnostopoulos, {\it ``Unitary matrix models: A Study of the string equation''}.


%-------------------------------------------------------------------------
% NewRecord: 73. NoArXiv Anagnostopoulos:1992um SlacID: 346491 DOI: NoDOI Citations:  0 ( 0 ) Pages: 12
\item Konstantinos N. Anagnostopoulos, Mark J. Bowick, {\it ``Multicriticality, scaling operators and mKdV flows for the symmetric unitary one matrix models''}.


%-------------------------------------------------------------------------
% NewRecord: 74. hep-th/9203005 Anagnostopoulos:1992bc SlacID: 332805 DOI: NoDOI Citations:  1 ( 1 ) Pages: 19
\item Konstantinos N. Anagnostopoulos, Mark J. Bowick, {\it ``Unitary one matrix models: String equation and flows''}, \href{https://arxiv.org/abs/hep-th/9203005}{arXiv:hep-th/9203005}.
\\\href{https://inspirehep.net/literature/?q=refersto%3Arecid%3A332805}{Cited by 1 (1)} articles

We review the Symmetric Unitary One Matrix Models. In particular we discuss the string equation in the operator formalism, the mKdV flows and the Virasoro Constraints. We focus on the t-function formalism for the flows and we describe its connection to the (big cell of the) Sato Grassmannian Gr via the Plucker embedding of Gr into a fermionic Fock space. Then the space of solutions to the string equation is an explicitly computable subspace of GrtimesGr which is invariant under the flows. (Invited talk delivered by M. J. Bowick at the Vth Regional Conference on Mathematical Physics, Edirne, Turkey; December 15-22, 1991.)
%-------------------------------------------------------------------------
% NewRecord: 75. hep-th/9112066 Anagnostopoulos:1991ss SlacID: 322308 DOI: 10.1007/BF02096545 Citations:  8 ( 7 ) Pages: 23
\item Konstantinos N. Anagnostopoulos, Mark J. Bowick, Albert S. Schwarz, {\it ``The Solution space of the unitary matrix model string equation and the Sato Grassmannian''}, \href{https://www.doi.org/10.1007/BF02096545}{Commun.Math.Phys. {\bf 148} (1992) } \href{https://arxiv.org/abs/hep-th/9112066}{[arXiv:hep-th/9112066]}, \href{https://www.doi.org/10.1007/BF02096545}{[doi:10.1007/BF02096545]}.
\\\href{https://inspirehep.net/literature/?q=refersto%3Arecid%3A322308}{Cited by 8 (7)} articles

The space of all solutions to the string equation of the symmetric unitary one-matrix model is determined. It is shown that the string equation is equivalent to simple conditions on points V1 and V2 in the big cell Gr of the Sato Grassmannian Gr. This is a consequence of a well-defined continuum limit in which the string equation has the simple form lb cp ,cq- rb =hbox{rm 1}, with cp and cq- 2times 2 matrices of differential operators. These conditions on V1 and V2 yield a simple system of first order differential equations whose analysis determines the space of all solutions to the string equation. This geometric formulation leads directly to the Virasoro constraints Ln,(ngeq 0), where Ln annihilate the two modified-KdV t-functions whose product gives the partition function of the Unitary Matrix Model.
%-------------------------------------------------------------------------
% NewRecord: 76. NoArXiv Anagnostopoulos:1991ym SlacID: 317536 DOI: 10.1142/S0217732391003183 Citations:  8 ( 4 ) Pages: 16
\item K.N. Anagnostopoulos, Mark J. Bowick, N. Ishibashi, {\it ``An Operator formalism for unitary matrix models''}, \href{https://www.doi.org/10.1142/S0217732391003183}{Mod.Phys.Lett.A {\bf 6} (1991) }, \href{https://www.doi.org/10.1142/S0217732391003183}{[doi:10.1142/S0217732391003183]}.
\\\href{https://inspirehep.net/literature/?q=refersto%3Arecid%3A317536}{Cited by 8 (4)} articles

We analyze the double scaling limit of unitary matrix models in terms of trigonometric orthogonal polynomials on the circle. In particular we find a compact formulation of the string equation at the kth multicritical point in terms of pseudodifferential operators and a corresponding action principle. We also relate this approach to the mKdV hierarchy which appears in the analysis in terms of conventional orthogonal polynomials on the circle.
\end{enumerate}
\end{document}
